%2multibyte Version: 5.50.0.2953 CodePage: 1252

\documentclass{article}
\usepackage{amsmath}

%%%%%%%%%%%%%%%%%%%%%%%%%%%%%%%%%%%%%%%%%%%%%%%%%%%%%%%%%%%%%%%%%%%%%%%%%%%%%%%%%%%%%%%%%%%%%%%%%%%%%%%%%%%%%%%%%%%%%%%%%%%%%%%%%%%%%%%%%%%%%%%%%%%%%%%%%%%%%%%%%%%%%%%%%%%%%%%%%%%%%%%%%%%%%%%%%%%%%%%%%%%%%%%%%%%%%%%%%%%%%%%%%%%%
%TCIDATA{OutputFilter=LATEX.DLL}
%TCIDATA{Version=5.50.0.2953}
%TCIDATA{Codepage=1252}
%TCIDATA{<META NAME="SaveForMode" CONTENT="1">}
%TCIDATA{BibliographyScheme=Manual}
%TCIDATA{Created=Tuesday, September 05, 2017 18:49:10}
%TCIDATA{LastRevised=Tuesday, September 12, 2017 12:28:13}
%TCIDATA{<META NAME="GraphicsSave" CONTENT="32">}
%TCIDATA{<META NAME="DocumentShell" CONTENT="Standard LaTeX\Blank - Standard LaTeX Article">}
%TCIDATA{CSTFile=40 LaTeX article.cst}

\newtheorem{theorem}{Theorem}
\newtheorem{acknowledgement}[theorem]{Acknowledgement}
\newtheorem{algorithm}[theorem]{Algorithm}
\newtheorem{axiom}[theorem]{Axiom}
\newtheorem{case}[theorem]{Case}
\newtheorem{claim}[theorem]{Claim}
\newtheorem{conclusion}[theorem]{Conclusion}
\newtheorem{condition}[theorem]{Condition}
\newtheorem{conjecture}[theorem]{Conjecture}
\newtheorem{corollary}[theorem]{Corollary}
\newtheorem{criterion}[theorem]{Criterion}
\newtheorem{definition}[theorem]{Definition}
\newtheorem{example}[theorem]{Example}
\newtheorem{exercise}[theorem]{Exercise}
\newtheorem{lemma}[theorem]{Lemma}
\newtheorem{notation}[theorem]{Notation}
\newtheorem{problem}[theorem]{Problem}
\newtheorem{proposition}[theorem]{Proposition}
\newtheorem{remark}[theorem]{Remark}
\newtheorem{solution}[theorem]{Solution}
\newtheorem{summary}[theorem]{Summary}
\newenvironment{proof}[1][Proof]{\noindent\textbf{#1.} }{\ \rule{0.5em}{0.5em}}
% Macros for Scientific Word 2.5 documents saved with the LaTeX filter.
%Copyright (C) 1994-96 TCI Software Research, Inc.
\typeout{TCILATEX Macros for Scientific Word 2.5 <04 SEP 96>.}

\typeout{NOTICE:  This macro file is NOT proprietary and may be 
freely copied and distributed.}


\makeatletter
%%
%% Changes
%% ** to \def\readFRAMEparams
%%    replaces h by H, if the float package is loaded
%%
\@ifundefined{@HHfloat}{\relax}{\typeout{** TCILaTeX detected 'float'-package:}	}	
%%see changes
%
%%%%%%%%%%%%%%%%%%%%%%
% macros for time
\newcount\@hour\newcount\@minute\chardef\@x10\chardef\@xv60
\def\tcitime{
\def\@time{%
  \@minute\time\@hour\@minute\divide\@hour\@xv
  \ifnum\@hour<\@x 0\fi\the\@hour:%
  \multiply\@hour\@xv\advance\@minute-\@hour
  \ifnum\@minute<\@x 0\fi\the\@minute
  }}%

%%%%%%%%%%%%%%%%%%%%%%
% macro for hyperref
\@ifundefined{hyperref}{\def\hyperref#1#2#3#4{#2\ref{#4}#3}}{}

% macro for external program call
\@ifundefined{qExtProgCall}{\def\qExtProgCall#1#2#3#4#5#6{\relax}}{}
%%%%%%%%%%%%%%%%%%%%%%
%
% macros for graphics
%
\def\FILENAME#1{#1}%
%
\def\QCTOpt[#1]#2{%
  \def\QCTOptB{#1}
  \def\QCTOptA{#2}
}
\def\QCTNOpt#1{%
  \def\QCTOptA{#1}
  \let\QCTOptB\empty
}
\def\Qct{%
  \@ifnextchar[{%
    \QCTOpt}{\QCTNOpt}
}
\def\QCBOpt[#1]#2{%
  \def\QCBOptB{#1}
  \def\QCBOptA{#2}
}
\def\QCBNOpt#1{%
  \def\QCBOptA{#1}
  \let\QCBOptB\empty
}
\def\Qcb{%
  \@ifnextchar[{%
    \QCBOpt}{\QCBNOpt}
}
\def\PrepCapArgs{%
  \ifx\QCBOptA\empty
    \ifx\QCTOptA\empty
      {}%
    \else
      \ifx\QCTOptB\empty
        {\QCTOptA}%
      \else
        [\QCTOptB]{\QCTOptA}%
      \fi
    \fi
  \else
    \ifx\QCBOptA\empty
      {}%
    \else
      \ifx\QCBOptB\empty
        {\QCBOptA}%
      \else
        [\QCBOptB]{\QCBOptA}%
      \fi
    \fi
  \fi
}
\newcount\GRAPHICSTYPE
%\GRAPHICSTYPE 0 is for TurboTeX
%\GRAPHICSTYPE 1 is for DVIWindo (PostScript)
%%%(removed)%\GRAPHICSTYPE 2 is for psfig (PostScript)
\GRAPHICSTYPE=\z@
\def\GRAPHICSPS#1{%
 \ifcase\GRAPHICSTYPE%\GRAPHICSTYPE=0
   \special{ps: #1}%
 \or%\GRAPHICSTYPE=1
   \special{language "PS", include "#1"}%
%%%\or%\GRAPHICSTYPE=2
%%%  #1%
 \fi
}%
%
\def\GRAPHICSHP#1{\special{include #1}}%
%
% \graffile{ body }                                  %#1
%          { contentswidth (scalar)  }               %#2
%          { contentsheight (scalar) }               %#3
%          { vertical shift when in-line (scalar) }  %#4
\def\graffile#1#2#3#4{%
%%% \ifnum\GRAPHICSTYPE=\tw@
%%%  %Following if using psfig
%%%  \@ifundefined{psfig}{\input psfig.tex}{}%
%%%  \psfig{file=#1, height=#3, width=#2}%
%%% \else
  %Following for all others
  % JCS - added BOXTHEFRAME, see below
    \leavevmode
    \raise -#4 \BOXTHEFRAME{%
        \hbox to #2{\raise #3\hbox to #2{\null #1\hfil}}}%
}%
%
% A box for drafts
\def\draftbox#1#2#3#4{%
 \leavevmode\raise -#4 \hbox{%
  \frame{\rlap{\protect\tiny #1}\hbox to #2%
   {\vrule height#3 width\z@ depth\z@\hfil}%
  }%
 }%
}%
%
\newcount\draft
\draft=\z@
\let\nographics=\draft
\newif\ifwasdraft
\wasdraftfalse

%  \GRAPHIC{ body }                                  %#1
%          { draft name }                            %#2
%          { contentswidth (scalar)  }               %#3
%          { contentsheight (scalar) }               %#4
%          { vertical shift when in-line (scalar) }  %#5
\def\GRAPHIC#1#2#3#4#5{%
 \ifnum\draft=\@ne\draftbox{#2}{#3}{#4}{#5}%
  \else\graffile{#1}{#3}{#4}{#5}%
  \fi
 }%
%
\def\addtoLaTeXparams#1{%
    \edef\LaTeXparams{\LaTeXparams #1}}%
%
% JCS -  added a switch BoxFrame that can 
% be set by including X in the frame params.
% If set a box is drawn around the frame.

\newif\ifBoxFrame \BoxFramefalse
\newif\ifOverFrame \OverFramefalse
\newif\ifUnderFrame \UnderFramefalse

\def\BOXTHEFRAME#1{%
   \hbox{%
      \ifBoxFrame
         \frame{#1}%
      \else
         {#1}%
      \fi
   }%
}


\def\doFRAMEparams#1{\BoxFramefalse\OverFramefalse\UnderFramefalse\readFRAMEparams#1\end}%
\def\readFRAMEparams#1{%
   \ifx#1\end%
  \let\next=\relax
  \else
  \ifx#1i\dispkind=\z@\fi
  \ifx#1d\dispkind=\@ne\fi
  \ifx#1f\dispkind=\tw@\fi
 	%% BEGIN CHANGES 0.12
	\ifx#1h
    \ifnum\dispkind=\tw@
			\@ifundefined{@HHfloat}{
			  \addtoLaTeXparams{h}
		 	 }{
         \def\LaTeXparams{H}
         \typeout{tcilatex: attribute align pos of FRAME  set to H}
         \typeout{\space \space \space \space all other placement options (tbp) are ignored }
   		 }
	  \else
			\addtoLaTeXparams{h}
    \fi
	\fi
  \if\LaTeXparams H
  	 \ifx#1t\fi	 %% ignore	all other placement
  	 \ifx#1b\fi	 %% options (tbp) 
     \ifx#1p\fi
  \else
      \ifx#1t\addtoLaTeXparams{t}\fi
      \ifx#1b\addtoLaTeXparams{b}\fi
      \ifx#1p\addtoLaTeXparams{p}\fi
  \fi
	%\typeout{LaTeXparms: \LaTeXparams}
%%END CHANGES 0.12

  \ifx#1X\BoxFrametrue\fi
  \ifx#1O\OverFrametrue\fi
  \ifx#1U\UnderFrametrue\fi
  \ifx#1w
    \ifnum\draft=1\wasdrafttrue\else\wasdraftfalse\fi
    \draft=\@ne
  \fi
  \let\next=\readFRAMEparams
  \fi
 \next
 }%
%
%Macro for In-line graphics object
%   \IFRAME{ contentswidth (scalar)  }               %#1
%          { contentsheight (scalar) }               %#2
%          { vertical shift when in-line (scalar) }  %#3
%          { draft name }                            %#4
%          { body }                                  %#5
%          { caption}                                %#6


\def\IFRAME#1#2#3#4#5#6{%
      \bgroup
      \let\QCTOptA\empty
      \let\QCTOptB\empty
      \let\QCBOptA\empty
      \let\QCBOptB\empty
      #6%
      \parindent=0pt%
      \leftskip=0pt
      \rightskip=0pt
      \setbox0 = \hbox{\QCBOptA}%
      \@tempdima = #1\relax
      \ifOverFrame
          % Do this later
          \typeout{This is not implemented yet}%
          \show\HELP
      \else
         \ifdim\wd0>\@tempdima
            \advance\@tempdima by \@tempdima
            \ifdim\wd0 >\@tempdima
               \textwidth=\@tempdima
               \setbox1 =\vbox{%
                  \noindent\hbox to \@tempdima{\hfill\GRAPHIC{#5}{#4}{#1}{#2}{#3}\hfill}\\%
                  \noindent\hbox to \@tempdima{\parbox[b]{\@tempdima}{\QCBOptA}}%
               }%
               \wd1=\@tempdima
            \else
               \textwidth=\wd0
               \setbox1 =\vbox{%
                 \noindent\hbox to \wd0{\hfill\GRAPHIC{#5}{#4}{#1}{#2}{#3}\hfill}\\%
                 \noindent\hbox{\QCBOptA}%
               }%
               \wd1=\wd0
            \fi
         \else
            %\show\BBB
            \ifdim\wd0>0pt
              \hsize=\@tempdima
              \setbox1 =\vbox{%
                \unskip\GRAPHIC{#5}{#4}{#1}{#2}{0pt}%
                \break
                \unskip\hbox to \@tempdima{\hfill \QCBOptA\hfill}%
              }%
              \wd1=\@tempdima
           \else
              \hsize=\@tempdima
              \setbox1 =\vbox{%
                \unskip\GRAPHIC{#5}{#4}{#1}{#2}{0pt}%
              }%
              \wd1=\@tempdima
           \fi
         \fi
         \@tempdimb=\ht1
         \advance\@tempdimb by \dp1
         \advance\@tempdimb by -#2%
         \advance\@tempdimb by #3%
         \leavevmode
         \raise -\@tempdimb \hbox{\box1}%
      \fi
      \egroup%
}%
%
%Macro for Display graphics object
%   \DFRAME{ contentswidth (scalar)  }               %#1
%          { contentsheight (scalar) }               %#2
%          { draft label }                           %#3
%          { name }                                  %#4
%          { caption}                                %#5
\def\DFRAME#1#2#3#4#5{%
 \begin{center}
     \let\QCTOptA\empty
     \let\QCTOptB\empty
     \let\QCBOptA\empty
     \let\QCBOptB\empty
     \ifOverFrame 
        #5\QCTOptA\par
     \fi
     \GRAPHIC{#4}{#3}{#1}{#2}{\z@}
     \ifUnderFrame 
        \nobreak\par #5\QCBOptA
     \fi
 \end{center}%
 }%
%
%Macro for Floating graphic object
%   \FFRAME{ framedata f|i tbph x F|T }              %#1
%          { contentswidth (scalar)  }               %#2
%          { contentsheight (scalar) }               %#3
%          { caption }                               %#4
%          { label }                                 %#5
%          { draft name }                            %#6
%          { body }                                  %#7
\def\FFRAME#1#2#3#4#5#6#7{%
 \begin{figure}[#1]%
  \let\QCTOptA\empty
  \let\QCTOptB\empty
  \let\QCBOptA\empty
  \let\QCBOptB\empty
  \ifOverFrame
    #4
    \ifx\QCTOptA\empty
    \else
      \ifx\QCTOptB\empty
        \caption{\QCTOptA}%
      \else
        \caption[\QCTOptB]{\QCTOptA}%
      \fi
    \fi
    \ifUnderFrame\else
      \label{#5}%
    \fi
  \else
    \UnderFrametrue%
  \fi
  \begin{center}\GRAPHIC{#7}{#6}{#2}{#3}{\z@}\end{center}%
  \ifUnderFrame
    #4
    \ifx\QCBOptA\empty
      \caption{}%
    \else
      \ifx\QCBOptB\empty
        \caption{\QCBOptA}%
      \else
        \caption[\QCBOptB]{\QCBOptA}%
      \fi
    \fi
    \label{#5}%
  \fi
  \end{figure}%
 }%
%
%
%    \FRAME{ framedata f|i tbph x F|T }              %#1
%          { contentswidth (scalar)  }               %#2
%          { contentsheight (scalar) }               %#3
%          { vertical shift when in-line (scalar) }  %#4
%          { caption }                               %#5
%          { label }                                 %#6
%          { name }                                  %#7
%          { body }                                  %#8
%
%    framedata is a string which can contain the following
%    characters: idftbphxFT
%    Their meaning is as follows:
%             i, d or f : in-line, display, or floating
%             t,b,p,h   : LaTeX floating placement options
%             x         : fit contents box to contents
%             F or T    : Figure or Table. 
%                         Later this can expand
%                         to a more general float class.
%
%
\newcount\dispkind%

\def\makeactives{
  \catcode`\"=\active
  \catcode`\;=\active
  \catcode`\:=\active
  \catcode`\'=\active
  \catcode`\~=\active
}
\bgroup
   \makeactives
   \gdef\activesoff{%
      \def"{\string"}
      \def;{\string;}
      \def:{\string:}
      \def'{\string'}
      \def~{\string~}
      %\bbl@deactivate{"}%
      %\bbl@deactivate{;}%
      %\bbl@deactivate{:}%
      %\bbl@deactivate{'}%
    }
\egroup

\def\FRAME#1#2#3#4#5#6#7#8{%
 \bgroup
 \@ifundefined{bbl@deactivate}{}{\activesoff}
 \ifnum\draft=\@ne
   \wasdrafttrue
 \else
   \wasdraftfalse%
 \fi
 \def\LaTeXparams{}%
 \dispkind=\z@
 \def\LaTeXparams{}%
 \doFRAMEparams{#1}%
 \ifnum\dispkind=\z@\IFRAME{#2}{#3}{#4}{#7}{#8}{#5}\else
  \ifnum\dispkind=\@ne\DFRAME{#2}{#3}{#7}{#8}{#5}\else
   \ifnum\dispkind=\tw@
    \edef\@tempa{\noexpand\FFRAME{\LaTeXparams}}%
    \@tempa{#2}{#3}{#5}{#6}{#7}{#8}%
    \fi
   \fi
  \fi
  \ifwasdraft\draft=1\else\draft=0\fi{}%
  \egroup
 }%
%
% This macro added to let SW gobble a parameter that
% should not be passed on and expanded. 

\def\TEXUX#1{"texux"}

%
% Macros for text attributes:
%
\def\BF#1{{\bf {#1}}}%
\def\NEG#1{\leavevmode\hbox{\rlap{\thinspace/}{$#1$}}}%
%
%%%%%%%%%%%%%%%%%%%%%%%%%%%%%%%%%%%%%%%%%%%%%%%%%%%%%%%%%%%%%%%%%%%%%%%%
%
%
% macros for user - defined functions
\def\func#1{\mathop{\rm #1}}%
\def\limfunc#1{\mathop{\rm #1}}%

%
% miscellaneous 
%\long\def\QQQ#1#2{}%
\long\def\QQQ#1#2{%
     \long\expandafter\def\csname#1\endcsname{#2}}%
%\def\QTP#1{}% JCS - this was changed becuase style editor will define QTP
\@ifundefined{QTP}{\def\QTP#1{}}{}
\@ifundefined{QEXCLUDE}{\def\QEXCLUDE#1{}}{}
%\@ifundefined{Qcb}{\def\Qcb#1{#1}}{}
%\@ifundefined{Qct}{\def\Qct#1{#1}}{}
\@ifundefined{Qlb}{\def\Qlb#1{#1}}{}
\@ifundefined{Qlt}{\def\Qlt#1{#1}}{}
\def\QWE{}%
\long\def\QQA#1#2{}%
%\def\QTR#1#2{{\em #2}}% Always \em!!!
%\def\QTR#1#2{\mbox{\begin{#1}#2\end{#1}}}%cb%%%
\def\QTR#1#2{{\csname#1\endcsname #2}}%(gp) Is this the best?
\long\def\TeXButton#1#2{#2}%
\long\def\QSubDoc#1#2{#2}%
\def\EXPAND#1[#2]#3{}%
\def\NOEXPAND#1[#2]#3{}%
\def\PROTECTED{}%
\def\LaTeXparent#1{}%
\def\ChildStyles#1{}%
\def\ChildDefaults#1{}%
\def\QTagDef#1#2#3{}%
%
% Macros for style editor docs
\@ifundefined{StyleEditBeginDoc}{\def\StyleEditBeginDoc{\relax}}{}
%
% Macros for footnotes
\def\QQfnmark#1{\footnotemark}
\def\QQfntext#1#2{\addtocounter{footnote}{#1}\footnotetext{#2}}
%
% Macros for indexing.
\def\MAKEINDEX{\makeatletter\input gnuindex.sty\makeatother\makeindex}%	
\@ifundefined{INDEX}{\def\INDEX#1#2{}{}}{}%
\@ifundefined{SUBINDEX}{\def\SUBINDEX#1#2#3{}{}{}}{}%
\@ifundefined{initial}%  
   {\def\initial#1{\bigbreak{\raggedright\large\bf #1}\kern 2\p@\penalty3000}}%
   {}%
\@ifundefined{entry}{\def\entry#1#2{\item {#1}, #2}}{}%
\@ifundefined{primary}{\def\primary#1{\item {#1}}}{}%
\@ifundefined{secondary}{\def\secondary#1#2{\subitem {#1}, #2}}{}%
%
%
\@ifundefined{ZZZ}{}{\MAKEINDEX\makeatletter}%
%
% Attempts to avoid problems with other styles
\@ifundefined{abstract}{%
 \def\abstract{%
  \if@twocolumn
   \section*{Abstract (Not appropriate in this style!)}%
   \else \small 
   \begin{center}{\bf Abstract\vspace{-.5em}\vspace{\z@}}\end{center}%
   \quotation 
   \fi
  }%
 }{%
 }%
\@ifundefined{endabstract}{\def\endabstract
  {\if@twocolumn\else\endquotation\fi}}{}%
\@ifundefined{maketitle}{\def\maketitle#1{}}{}%
\@ifundefined{affiliation}{\def\affiliation#1{}}{}%
\@ifundefined{proof}{\def\proof{\noindent{\bfseries Proof. }}}{}%
\@ifundefined{endproof}{\def\endproof{\mbox{\ \rule{.1in}{.1in}}}}{}%
\@ifundefined{newfield}{\def\newfield#1#2{}}{}%
\@ifundefined{chapter}{\def\chapter#1{\par(Chapter head:)#1\par }%
 \newcount\c@chapter}{}%
\@ifundefined{part}{\def\part#1{\par(Part head:)#1\par }}{}%
\@ifundefined{section}{\def\section#1{\par(Section head:)#1\par }}{}%
\@ifundefined{subsection}{\def\subsection#1%
 {\par(Subsection head:)#1\par }}{}%
\@ifundefined{subsubsection}{\def\subsubsection#1%
 {\par(Subsubsection head:)#1\par }}{}%
\@ifundefined{paragraph}{\def\paragraph#1%
 {\par(Subsubsubsection head:)#1\par }}{}%
\@ifundefined{subparagraph}{\def\subparagraph#1%
 {\par(Subsubsubsubsection head:)#1\par }}{}%
%%%%%%%%%%%%%%%%%%%%%%%%%%%%%%%%%%%%%%%%%%%%%%%%%%%%%%%%%%%%%%%%%%%%%%%%
% These symbols are not recognized by LaTeX
\@ifundefined{therefore}{\def\therefore{}}{}%
\@ifundefined{backepsilon}{\def\backepsilon{}}{}%
\@ifundefined{yen}{\def\yen{\hbox{\rm\rlap=Y}}}{}%
\@ifundefined{registered}{%
   \def\registered{\relax\ifmmode{}\r@gistered
                    \else$\m@th\r@gistered$\fi}%
 \def\r@gistered{^{\ooalign
  {\hfil\raise.07ex\hbox{$\scriptstyle\rm\text{R}$}\hfil\crcr
  \mathhexbox20D}}}}{}%
\@ifundefined{Eth}{\def\Eth{}}{}%
\@ifundefined{eth}{\def\eth{}}{}%
\@ifundefined{Thorn}{\def\Thorn{}}{}%
\@ifundefined{thorn}{\def\thorn{}}{}%
% A macro to allow any symbol that requires math to appear in text
\def\TEXTsymbol#1{\mbox{$#1$}}%
\@ifundefined{degree}{\def\degree{{}^{\circ}}}{}%
%
% macros for T3TeX files
\newdimen\theight
\def\Column{%
 \vadjust{\setbox\z@=\hbox{\scriptsize\quad\quad tcol}%
  \theight=\ht\z@\advance\theight by \dp\z@\advance\theight by \lineskip
  \kern -\theight \vbox to \theight{%
   \rightline{\rlap{\box\z@}}%
   \vss
   }%
  }%
 }%
%
\def\qed{%
 \ifhmode\unskip\nobreak\fi\ifmmode\ifinner\else\hskip5\p@\fi\fi
 \hbox{\hskip5\p@\vrule width4\p@ height6\p@ depth1.5\p@\hskip\p@}%
 }%
%
\def\cents{\hbox{\rm\rlap/c}}%
\def\miss{\hbox{\vrule height2\p@ width 2\p@ depth\z@}}%
%\def\miss{\hbox{.}}%        %another possibility 
%
\def\vvert{\Vert}%           %always translated to \left| or \right|
%
\def\tcol#1{{\baselineskip=6\p@ \vcenter{#1}} \Column}  %
%
\def\dB{\hbox{{}}}%                 %dummy entry in column 
\def\mB#1{\hbox{$#1$}}%             %column entry
\def\nB#1{\hbox{#1}}%               %column entry (not math)
%
%\newcount\notenumber
%\def\clearnotenumber{\notenumber=0}
%\def\note{\global\advance\notenumber by 1
% \footnote{$^{\the\notenumber}$}}
%\def\note{\global\advance\notenumber by 1
\def\note{$^{\dag}}%
%
%

\def\newfmtname{LaTeX2e}
\def\chkcompat{%
   \if@compatibility
   \else
     \usepackage{latexsym}
   \fi
}

\ifx\fmtname\newfmtname
  \DeclareOldFontCommand{\rm}{\normalfont\rmfamily}{\mathrm}
  \DeclareOldFontCommand{\sf}{\normalfont\sffamily}{\mathsf}
  \DeclareOldFontCommand{\tt}{\normalfont\ttfamily}{\mathtt}
  \DeclareOldFontCommand{\bf}{\normalfont\bfseries}{\mathbf}
  \DeclareOldFontCommand{\it}{\normalfont\itshape}{\mathit}
  \DeclareOldFontCommand{\sl}{\normalfont\slshape}{\@nomath\sl}
  \DeclareOldFontCommand{\sc}{\normalfont\scshape}{\@nomath\sc}
  \chkcompat
\fi

%
% Greek bold macros
% Redefine all of the math symbols 
% which might be bolded	 - there are 
% probably others to add to this list

\def\alpha{{\Greekmath 010B}}%
\def\beta{{\Greekmath 010C}}%
\def\gamma{{\Greekmath 010D}}%
\def\delta{{\Greekmath 010E}}%
\def\epsilon{{\Greekmath 010F}}%
\def\zeta{{\Greekmath 0110}}%
\def\eta{{\Greekmath 0111}}%
\def\theta{{\Greekmath 0112}}%
\def\iota{{\Greekmath 0113}}%
\def\kappa{{\Greekmath 0114}}%
\def\lambda{{\Greekmath 0115}}%
\def\mu{{\Greekmath 0116}}%
\def\nu{{\Greekmath 0117}}%
\def\xi{{\Greekmath 0118}}%
\def\pi{{\Greekmath 0119}}%
\def\rho{{\Greekmath 011A}}%
\def\sigma{{\Greekmath 011B}}%
\def\tau{{\Greekmath 011C}}%
\def\upsilon{{\Greekmath 011D}}%
\def\phi{{\Greekmath 011E}}%
\def\chi{{\Greekmath 011F}}%
\def\psi{{\Greekmath 0120}}%
\def\omega{{\Greekmath 0121}}%
\def\varepsilon{{\Greekmath 0122}}%
\def\vartheta{{\Greekmath 0123}}%
\def\varpi{{\Greekmath 0124}}%
\def\varrho{{\Greekmath 0125}}%
\def\varsigma{{\Greekmath 0126}}%
\def\varphi{{\Greekmath 0127}}%

\def\nabla{{\Greekmath 0272}}
\def\FindBoldGroup{%
   {\setbox0=\hbox{$\mathbf{x\global\edef\theboldgroup{\the\mathgroup}}$}}%
}

\def\Greekmath#1#2#3#4{%
    \if@compatibility
        \ifnum\mathgroup=\symbold
           \mathchoice{\mbox{\boldmath$\displaystyle\mathchar"#1#2#3#4$}}%
                      {\mbox{\boldmath$\textstyle\mathchar"#1#2#3#4$}}%
                      {\mbox{\boldmath$\scriptstyle\mathchar"#1#2#3#4$}}%
                      {\mbox{\boldmath$\scriptscriptstyle\mathchar"#1#2#3#4$}}%
        \else
           \mathchar"#1#2#3#4% 
        \fi 
    \else 
        \FindBoldGroup
        \ifnum\mathgroup=\theboldgroup % For 2e
           \mathchoice{\mbox{\boldmath$\displaystyle\mathchar"#1#2#3#4$}}%
                      {\mbox{\boldmath$\textstyle\mathchar"#1#2#3#4$}}%
                      {\mbox{\boldmath$\scriptstyle\mathchar"#1#2#3#4$}}%
                      {\mbox{\boldmath$\scriptscriptstyle\mathchar"#1#2#3#4$}}%
        \else
           \mathchar"#1#2#3#4% 
        \fi     	    
	  \fi}

\newif\ifGreekBold  \GreekBoldfalse
\let\SAVEPBF=\pbf
\def\pbf{\GreekBoldtrue\SAVEPBF}%
%

\@ifundefined{theorem}{\newtheorem{theorem}{Theorem}}{}
\@ifundefined{lemma}{\newtheorem{lemma}[theorem]{Lemma}}{}
\@ifundefined{corollary}{\newtheorem{corollary}[theorem]{Corollary}}{}
\@ifundefined{conjecture}{\newtheorem{conjecture}[theorem]{Conjecture}}{}
\@ifundefined{proposition}{\newtheorem{proposition}[theorem]{Proposition}}{}
\@ifundefined{axiom}{\newtheorem{axiom}{Axiom}}{}
\@ifundefined{remark}{\newtheorem{remark}{Remark}}{}
\@ifundefined{example}{\newtheorem{example}{Example}}{}
\@ifundefined{exercise}{\newtheorem{exercise}{Exercise}}{}
\@ifundefined{definition}{\newtheorem{definition}{Definition}}{}


\@ifundefined{mathletters}{%
  %\def\theequation{\arabic{equation}}
  \newcounter{equationnumber}  
  \def\mathletters{%
     \addtocounter{equation}{1}
     \edef\@currentlabel{\theequation}%
     \setcounter{equationnumber}{\c@equation}
     \setcounter{equation}{0}%
     \edef\theequation{\@currentlabel\noexpand\alph{equation}}%
  }
  \def\endmathletters{%
     \setcounter{equation}{\value{equationnumber}}%
  }
}{}

%Logos
\@ifundefined{BibTeX}{%
    \def\BibTeX{{\rm B\kern-.05em{\sc i\kern-.025em b}\kern-.08em
                 T\kern-.1667em\lower.7ex\hbox{E}\kern-.125emX}}}{}%
\@ifundefined{AmS}%
    {\def\AmS{{\protect\usefont{OMS}{cmsy}{m}{n}%
                A\kern-.1667em\lower.5ex\hbox{M}\kern-.125emS}}}{}%
\@ifundefined{AmSTeX}{\def\AmSTeX{\protect\AmS-\protect\TeX\@}}{}%
%

%%%%%%%%%%%%%%%%%%%%%%%%%%%%%%%%%%%%%%%%%%%%%%%%%%%%%%%%%%%%%%%%%%%%%%%
% NOTE: The rest of this file is read only if amstex has not been
% loaded.  This section is used to define amstex constructs in the
% event they have not been defined.
%
%
\ifx\ds@amstex\relax
   \message{amstex already loaded}\makeatother\endinput% 2.09 compatability
\else
   \@ifpackageloaded{amstex}%
      {\message{amstex already loaded}\makeatother\endinput}
      {}
   \@ifpackageloaded{amsgen}%
      {\message{amsgen already loaded}\makeatother\endinput}
      {}
\fi
%%%%%%%%%%%%%%%%%%%%%%%%%%%%%%%%%%%%%%%%%%%%%%%%%%%%%%%%%%%%%%%%%%%%%%%%
%%
%
%
%  Macros to define some AMS LaTeX constructs when 
%  AMS LaTeX has not been loaded
% 
% These macros are copied from the AMS-TeX package for doing
% multiple integrals.
%
\def\DN@{\def\next@}%
\def\eat@#1{}%
\let\DOTSI\relax
\def\RIfM@{\relax\ifmmode}%
\def\FN@{\futurelet\next}%
\newcount\intno@
\def\iint{\DOTSI\intno@\tw@\FN@\ints@}%
\def\iiint{\DOTSI\intno@\thr@@\FN@\ints@}%
\def\iiiint{\DOTSI\intno@4 \FN@\ints@}%
\def\idotsint{\DOTSI\intno@\z@\FN@\ints@}%
\def\ints@{\findlimits@\ints@@}%
\newif\iflimtoken@
\newif\iflimits@
\def\findlimits@{\limtoken@true\ifx\next\limits\limits@true
 \else\ifx\next\nolimits\limits@false\else
 \limtoken@false\ifx\ilimits@\nolimits\limits@false\else
 \ifinner\limits@false\else\limits@true\fi\fi\fi\fi}%
\def\multint@{\int\ifnum\intno@=\z@\intdots@                          %1
 \else\intkern@\fi                                                    %2
 \ifnum\intno@>\tw@\int\intkern@\fi                                   %3
 \ifnum\intno@>\thr@@\int\intkern@\fi                                 %4
 \int}%                                                               %5
\def\multintlimits@{\intop\ifnum\intno@=\z@\intdots@\else\intkern@\fi
 \ifnum\intno@>\tw@\intop\intkern@\fi
 \ifnum\intno@>\thr@@\intop\intkern@\fi\intop}%
\def\intic@{%
    \mathchoice{\hskip.5em}{\hskip.4em}{\hskip.4em}{\hskip.4em}}%
\def\negintic@{\mathchoice
 {\hskip-.5em}{\hskip-.4em}{\hskip-.4em}{\hskip-.4em}}%
\def\ints@@{\iflimtoken@                                              %1
 \def\ints@@@{\iflimits@\negintic@
   \mathop{\intic@\multintlimits@}\limits                             %2
  \else\multint@\nolimits\fi                                          %3
  \eat@}%                                                             %4
 \else                                                                %5
 \def\ints@@@{\iflimits@\negintic@
  \mathop{\intic@\multintlimits@}\limits\else
  \multint@\nolimits\fi}\fi\ints@@@}%
\def\intkern@{\mathchoice{\!\!\!}{\!\!}{\!\!}{\!\!}}%
\def\plaincdots@{\mathinner{\cdotp\cdotp\cdotp}}%
\def\intdots@{\mathchoice{\plaincdots@}%
 {{\cdotp}\mkern1.5mu{\cdotp}\mkern1.5mu{\cdotp}}%
 {{\cdotp}\mkern1mu{\cdotp}\mkern1mu{\cdotp}}%
 {{\cdotp}\mkern1mu{\cdotp}\mkern1mu{\cdotp}}}%
%
%
%  These macros are for doing the AMS \text{} construct
%
\def\RIfM@{\relax\protect\ifmmode}
\def\text{\RIfM@\expandafter\text@\else\expandafter\mbox\fi}
\let\nfss@text\text
\def\text@#1{\mathchoice
   {\textdef@\displaystyle\f@size{#1}}%
   {\textdef@\textstyle\tf@size{\firstchoice@false #1}}%
   {\textdef@\textstyle\sf@size{\firstchoice@false #1}}%
   {\textdef@\textstyle \ssf@size{\firstchoice@false #1}}%
   \glb@settings}

\def\textdef@#1#2#3{\hbox{{%
                    \everymath{#1}%
                    \let\f@size#2\selectfont
                    #3}}}
\newif\iffirstchoice@
\firstchoice@true
%
%    Old Scheme for \text
%
%\def\rmfam{\z@}%
%\newif\iffirstchoice@
%\firstchoice@true
%\def\textfonti{\the\textfont\@ne}%
%\def\textfontii{\the\textfont\tw@}%
%\def\text{\RIfM@\expandafter\text@\else\expandafter\text@@\fi}%
%\def\text@@#1{\leavevmode\hbox{#1}}%
%\def\text@#1{\mathchoice
% {\hbox{\everymath{\displaystyle}\def\textfonti{\the\textfont\@ne}%
%  \def\textfontii{\the\textfont\tw@}\textdef@@ T#1}}%
% {\hbox{\firstchoice@false
%  \everymath{\textstyle}\def\textfonti{\the\textfont\@ne}%
%  \def\textfontii{\the\textfont\tw@}\textdef@@ T#1}}%
% {\hbox{\firstchoice@false
%  \everymath{\scriptstyle}\def\textfonti{\the\scriptfont\@ne}%
%  \def\textfontii{\the\scriptfont\tw@}\textdef@@ S\rm#1}}%
% {\hbox{\firstchoice@false
%  \everymath{\scriptscriptstyle}\def\textfonti
%  {\the\scriptscriptfont\@ne}%
%  \def\textfontii{\the\scriptscriptfont\tw@}\textdef@@ s\rm#1}}}%
%\def\textdef@@#1{\textdef@#1\rm\textdef@#1\bf\textdef@#1\sl
%    \textdef@#1\it}%
%\def\DN@{\def\next@}%
%\def\eat@#1{}%
%\def\textdef@#1#2{%
% \DN@{\csname\expandafter\eat@\string#2fam\endcsname}%
% \if S#1\edef#2{\the\scriptfont\next@\relax}%
% \else\if s#1\edef#2{\the\scriptscriptfont\next@\relax}%
% \else\edef#2{\the\textfont\next@\relax}\fi\fi}%
%
%
%These are the AMS constructs for multiline limits.
%
\def\Let@{\relax\iffalse{\fi\let\\=\cr\iffalse}\fi}%
\def\vspace@{\def\vspace##1{\crcr\noalign{\vskip##1\relax}}}%
\def\multilimits@{\bgroup\vspace@\Let@
 \baselineskip\fontdimen10 \scriptfont\tw@
 \advance\baselineskip\fontdimen12 \scriptfont\tw@
 \lineskip\thr@@\fontdimen8 \scriptfont\thr@@
 \lineskiplimit\lineskip
 \vbox\bgroup\ialign\bgroup\hfil$\m@th\scriptstyle{##}$\hfil\crcr}%
\def\Sb{_\multilimits@}%
\def\endSb{\crcr\egroup\egroup\egroup}%
\def\Sp{^\multilimits@}%
\let\endSp\endSb
%
%
%These are AMS constructs for horizontal arrows
%
\newdimen\ex@
\ex@.2326ex
\def\rightarrowfill@#1{$#1\m@th\mathord-\mkern-6mu\cleaders
 \hbox{$#1\mkern-2mu\mathord-\mkern-2mu$}\hfill
 \mkern-6mu\mathord\rightarrow$}%
\def\leftarrowfill@#1{$#1\m@th\mathord\leftarrow\mkern-6mu\cleaders
 \hbox{$#1\mkern-2mu\mathord-\mkern-2mu$}\hfill\mkern-6mu\mathord-$}%
\def\leftrightarrowfill@#1{$#1\m@th\mathord\leftarrow
\mkern-6mu\cleaders
 \hbox{$#1\mkern-2mu\mathord-\mkern-2mu$}\hfill
 \mkern-6mu\mathord\rightarrow$}%
\def\overrightarrow{\mathpalette\overrightarrow@}%
\def\overrightarrow@#1#2{\vbox{\ialign{##\crcr\rightarrowfill@#1\crcr
 \noalign{\kern-\ex@\nointerlineskip}$\m@th\hfil#1#2\hfil$\crcr}}}%
\let\overarrow\overrightarrow
\def\overleftarrow{\mathpalette\overleftarrow@}%
\def\overleftarrow@#1#2{\vbox{\ialign{##\crcr\leftarrowfill@#1\crcr
 \noalign{\kern-\ex@\nointerlineskip}$\m@th\hfil#1#2\hfil$\crcr}}}%
\def\overleftrightarrow{\mathpalette\overleftrightarrow@}%
\def\overleftrightarrow@#1#2{\vbox{\ialign{##\crcr
   \leftrightarrowfill@#1\crcr
 \noalign{\kern-\ex@\nointerlineskip}$\m@th\hfil#1#2\hfil$\crcr}}}%
\def\underrightarrow{\mathpalette\underrightarrow@}%
\def\underrightarrow@#1#2{\vtop{\ialign{##\crcr$\m@th\hfil#1#2\hfil
  $\crcr\noalign{\nointerlineskip}\rightarrowfill@#1\crcr}}}%
\let\underarrow\underrightarrow
\def\underleftarrow{\mathpalette\underleftarrow@}%
\def\underleftarrow@#1#2{\vtop{\ialign{##\crcr$\m@th\hfil#1#2\hfil
  $\crcr\noalign{\nointerlineskip}\leftarrowfill@#1\crcr}}}%
\def\underleftrightarrow{\mathpalette\underleftrightarrow@}%
\def\underleftrightarrow@#1#2{\vtop{\ialign{##\crcr$\m@th
  \hfil#1#2\hfil$\crcr
 \noalign{\nointerlineskip}\leftrightarrowfill@#1\crcr}}}%
%%%%%%%%%%%%%%%%%%%%%

% 94.0815 by Jon:

\def\qopnamewl@#1{\mathop{\operator@font#1}\nlimits@}
\let\nlimits@\displaylimits
\def\setboxz@h{\setbox\z@\hbox}


\def\varlim@#1#2{\mathop{\vtop{\ialign{##\crcr
 \hfil$#1\m@th\operator@font lim$\hfil\crcr
 \noalign{\nointerlineskip}#2#1\crcr
 \noalign{\nointerlineskip\kern-\ex@}\crcr}}}}

 \def\rightarrowfill@#1{\m@th\setboxz@h{$#1-$}\ht\z@\z@
  $#1\copy\z@\mkern-6mu\cleaders
  \hbox{$#1\mkern-2mu\box\z@\mkern-2mu$}\hfill
  \mkern-6mu\mathord\rightarrow$}
\def\leftarrowfill@#1{\m@th\setboxz@h{$#1-$}\ht\z@\z@
  $#1\mathord\leftarrow\mkern-6mu\cleaders
  \hbox{$#1\mkern-2mu\copy\z@\mkern-2mu$}\hfill
  \mkern-6mu\box\z@$}


\def\projlim{\qopnamewl@{proj\,lim}}
\def\injlim{\qopnamewl@{inj\,lim}}
\def\varinjlim{\mathpalette\varlim@\rightarrowfill@}
\def\varprojlim{\mathpalette\varlim@\leftarrowfill@}
\def\varliminf{\mathpalette\varliminf@{}}
\def\varliminf@#1{\mathop{\underline{\vrule\@depth.2\ex@\@width\z@
   \hbox{$#1\m@th\operator@font lim$}}}}
\def\varlimsup{\mathpalette\varlimsup@{}}
\def\varlimsup@#1{\mathop{\overline
  {\hbox{$#1\m@th\operator@font lim$}}}}

%
%%%%%%%%%%%%%%%%%%%%%%%%%%%%%%%%%%%%%%%%%%%%%%%%%%%%%%%%%%%%%%%%%%%%%
%
\def\tfrac#1#2{{\textstyle {#1 \over #2}}}%
\def\dfrac#1#2{{\displaystyle {#1 \over #2}}}%
\def\binom#1#2{{#1 \choose #2}}%
\def\tbinom#1#2{{\textstyle {#1 \choose #2}}}%
\def\dbinom#1#2{{\displaystyle {#1 \choose #2}}}%
\def\QATOP#1#2{{#1 \atop #2}}%
\def\QTATOP#1#2{{\textstyle {#1 \atop #2}}}%
\def\QDATOP#1#2{{\displaystyle {#1 \atop #2}}}%
\def\QABOVE#1#2#3{{#2 \above#1 #3}}%
\def\QTABOVE#1#2#3{{\textstyle {#2 \above#1 #3}}}%
\def\QDABOVE#1#2#3{{\displaystyle {#2 \above#1 #3}}}%
\def\QOVERD#1#2#3#4{{#3 \overwithdelims#1#2 #4}}%
\def\QTOVERD#1#2#3#4{{\textstyle {#3 \overwithdelims#1#2 #4}}}%
\def\QDOVERD#1#2#3#4{{\displaystyle {#3 \overwithdelims#1#2 #4}}}%
\def\QATOPD#1#2#3#4{{#3 \atopwithdelims#1#2 #4}}%
\def\QTATOPD#1#2#3#4{{\textstyle {#3 \atopwithdelims#1#2 #4}}}%
\def\QDATOPD#1#2#3#4{{\displaystyle {#3 \atopwithdelims#1#2 #4}}}%
\def\QABOVED#1#2#3#4#5{{#4 \abovewithdelims#1#2#3 #5}}%
\def\QTABOVED#1#2#3#4#5{{\textstyle 
   {#4 \abovewithdelims#1#2#3 #5}}}%
\def\QDABOVED#1#2#3#4#5{{\displaystyle 
   {#4 \abovewithdelims#1#2#3 #5}}}%
%
% Macros for text size operators:

%JCS - added braces and \mathop around \displaystyle\int, etc.
%
\def\tint{\mathop{\textstyle \int}}%
\def\tiint{\mathop{\textstyle \iint }}%
\def\tiiint{\mathop{\textstyle \iiint }}%
\def\tiiiint{\mathop{\textstyle \iiiint }}%
\def\tidotsint{\mathop{\textstyle \idotsint }}%
\def\toint{\mathop{\textstyle \oint}}%
\def\tsum{\mathop{\textstyle \sum }}%
\def\tprod{\mathop{\textstyle \prod }}%
\def\tbigcap{\mathop{\textstyle \bigcap }}%
\def\tbigwedge{\mathop{\textstyle \bigwedge }}%
\def\tbigoplus{\mathop{\textstyle \bigoplus }}%
\def\tbigodot{\mathop{\textstyle \bigodot }}%
\def\tbigsqcup{\mathop{\textstyle \bigsqcup }}%
\def\tcoprod{\mathop{\textstyle \coprod }}%
\def\tbigcup{\mathop{\textstyle \bigcup }}%
\def\tbigvee{\mathop{\textstyle \bigvee }}%
\def\tbigotimes{\mathop{\textstyle \bigotimes }}%
\def\tbiguplus{\mathop{\textstyle \biguplus }}%
%
%
%Macros for display size operators:
%

\def\dint{\mathop{\displaystyle \int}}%
\def\diint{\mathop{\displaystyle \iint }}%
\def\diiint{\mathop{\displaystyle \iiint }}%
\def\diiiint{\mathop{\displaystyle \iiiint }}%
\def\didotsint{\mathop{\displaystyle \idotsint }}%
\def\doint{\mathop{\displaystyle \oint}}%
\def\dsum{\mathop{\displaystyle \sum }}%
\def\dprod{\mathop{\displaystyle \prod }}%
\def\dbigcap{\mathop{\displaystyle \bigcap }}%
\def\dbigwedge{\mathop{\displaystyle \bigwedge }}%
\def\dbigoplus{\mathop{\displaystyle \bigoplus }}%
\def\dbigodot{\mathop{\displaystyle \bigodot }}%
\def\dbigsqcup{\mathop{\displaystyle \bigsqcup }}%
\def\dcoprod{\mathop{\displaystyle \coprod }}%
\def\dbigcup{\mathop{\displaystyle \bigcup }}%
\def\dbigvee{\mathop{\displaystyle \bigvee }}%
\def\dbigotimes{\mathop{\displaystyle \bigotimes }}%
\def\dbiguplus{\mathop{\displaystyle \biguplus }}%
%
%Companion to stackrel
\def\stackunder#1#2{\mathrel{\mathop{#2}\limits_{#1}}}%
%
%
% These are AMS environments that will be defined to
% be verbatims if amstex has not actually been 
% loaded
%
%
\begingroup \catcode `|=0 \catcode `[= 1
\catcode`]=2 \catcode `\{=12 \catcode `\}=12
\catcode`\\=12 
|gdef|@alignverbatim#1\end{align}[#1|end[align]]
|gdef|@salignverbatim#1\end{align*}[#1|end[align*]]

|gdef|@alignatverbatim#1\end{alignat}[#1|end[alignat]]
|gdef|@salignatverbatim#1\end{alignat*}[#1|end[alignat*]]

|gdef|@xalignatverbatim#1\end{xalignat}[#1|end[xalignat]]
|gdef|@sxalignatverbatim#1\end{xalignat*}[#1|end[xalignat*]]

|gdef|@gatherverbatim#1\end{gather}[#1|end[gather]]
|gdef|@sgatherverbatim#1\end{gather*}[#1|end[gather*]]

|gdef|@gatherverbatim#1\end{gather}[#1|end[gather]]
|gdef|@sgatherverbatim#1\end{gather*}[#1|end[gather*]]


|gdef|@multilineverbatim#1\end{multiline}[#1|end[multiline]]
|gdef|@smultilineverbatim#1\end{multiline*}[#1|end[multiline*]]

|gdef|@arraxverbatim#1\end{arrax}[#1|end[arrax]]
|gdef|@sarraxverbatim#1\end{arrax*}[#1|end[arrax*]]

|gdef|@tabulaxverbatim#1\end{tabulax}[#1|end[tabulax]]
|gdef|@stabulaxverbatim#1\end{tabulax*}[#1|end[tabulax*]]


|endgroup
  

  
\def\align{\@verbatim \frenchspacing\@vobeyspaces \@alignverbatim
You are using the "align" environment in a style in which it is not defined.}
\let\endalign=\endtrivlist
 
\@namedef{align*}{\@verbatim\@salignverbatim
You are using the "align*" environment in a style in which it is not defined.}
\expandafter\let\csname endalign*\endcsname =\endtrivlist




\def\alignat{\@verbatim \frenchspacing\@vobeyspaces \@alignatverbatim
You are using the "alignat" environment in a style in which it is not defined.}
\let\endalignat=\endtrivlist
 
\@namedef{alignat*}{\@verbatim\@salignatverbatim
You are using the "alignat*" environment in a style in which it is not defined.}
\expandafter\let\csname endalignat*\endcsname =\endtrivlist




\def\xalignat{\@verbatim \frenchspacing\@vobeyspaces \@xalignatverbatim
You are using the "xalignat" environment in a style in which it is not defined.}
\let\endxalignat=\endtrivlist
 
\@namedef{xalignat*}{\@verbatim\@sxalignatverbatim
You are using the "xalignat*" environment in a style in which it is not defined.}
\expandafter\let\csname endxalignat*\endcsname =\endtrivlist




\def\gather{\@verbatim \frenchspacing\@vobeyspaces \@gatherverbatim
You are using the "gather" environment in a style in which it is not defined.}
\let\endgather=\endtrivlist
 
\@namedef{gather*}{\@verbatim\@sgatherverbatim
You are using the "gather*" environment in a style in which it is not defined.}
\expandafter\let\csname endgather*\endcsname =\endtrivlist


\def\multiline{\@verbatim \frenchspacing\@vobeyspaces \@multilineverbatim
You are using the "multiline" environment in a style in which it is not defined.}
\let\endmultiline=\endtrivlist
 
\@namedef{multiline*}{\@verbatim\@smultilineverbatim
You are using the "multiline*" environment in a style in which it is not defined.}
\expandafter\let\csname endmultiline*\endcsname =\endtrivlist


\def\arrax{\@verbatim \frenchspacing\@vobeyspaces \@arraxverbatim
You are using a type of "array" construct that is only allowed in AmS-LaTeX.}
\let\endarrax=\endtrivlist

\def\tabulax{\@verbatim \frenchspacing\@vobeyspaces \@tabulaxverbatim
You are using a type of "tabular" construct that is only allowed in AmS-LaTeX.}
\let\endtabulax=\endtrivlist

 
\@namedef{arrax*}{\@verbatim\@sarraxverbatim
You are using a type of "array*" construct that is only allowed in AmS-LaTeX.}
\expandafter\let\csname endarrax*\endcsname =\endtrivlist

\@namedef{tabulax*}{\@verbatim\@stabulaxverbatim
You are using a type of "tabular*" construct that is only allowed in AmS-LaTeX.}
\expandafter\let\csname endtabulax*\endcsname =\endtrivlist

% macro to simulate ams tag construct


% This macro is a fix to eqnarray
\def\@@eqncr{\let\@tempa\relax
    \ifcase\@eqcnt \def\@tempa{& & &}\or \def\@tempa{& &}%
      \else \def\@tempa{&}\fi
     \@tempa
     \if@eqnsw
        \iftag@
           \@taggnum
        \else
           \@eqnnum\stepcounter{equation}%
        \fi
     \fi
     \global\tag@false
     \global\@eqnswtrue
     \global\@eqcnt\z@\cr}


% This macro is a fix to the equation environment
 \def\endequation{%
     \ifmmode\ifinner % FLEQN hack
      \iftag@
        \addtocounter{equation}{-1} % undo the increment made in the begin part
        $\hfil
           \displaywidth\linewidth\@taggnum\egroup \endtrivlist
        \global\tag@false
        \global\@ignoretrue   
      \else
        $\hfil
           \displaywidth\linewidth\@eqnnum\egroup \endtrivlist
        \global\tag@false
        \global\@ignoretrue 
      \fi
     \else   
      \iftag@
        \addtocounter{equation}{-1} % undo the increment made in the begin part
        \eqno \hbox{\@taggnum}
        \global\tag@false%
        $$\global\@ignoretrue
      \else
        \eqno \hbox{\@eqnnum}% $$ BRACE MATCHING HACK
        $$\global\@ignoretrue
      \fi
     \fi\fi
 } 

 \newif\iftag@ \tag@false
 
 \def\tag{\@ifnextchar*{\@tagstar}{\@tag}}
 \def\@tag#1{%
     \global\tag@true
     \global\def\@taggnum{(#1)}}
 \def\@tagstar*#1{%
     \global\tag@true
     \global\def\@taggnum{#1}%  
}

% Do not add anything to the end of this file.  
% The last section of the file is loaded only if 
% amstex has not been.



\makeatother
\endinput

\begin{document}


\section{Computing the gradient}

The loss function is given by 
\begin{eqnarray*}
L\left( \tilde{\theta}\right)  &=&E\left[ \sum_{k=1}^{n}\min \left(
x_{k},w_{k}\right) +\sum_{k=1}^{n}\left( 1+\gamma \right) \left(
x_{k}-w_{k}\right) u_{k}\left( x\right) \right]  \\
&&x\text{ drawn from joint cdf }F\left( x;\tilde{\theta}\right)  \\
u_{k}\left( x\right)  &=&\left\{ 
\begin{array}{ccc}
0 & , & \text{if }k\notin D \\ 
e_{k}\left( I_{D}-\left( 1+\gamma \right) A_{D}\right) ^{-1}1_{D} & , & 
\text{if }k\in D%
\end{array}%
\right.  \\
D &=&\left\{ i:p_{i}<\bar{p}_{i}\right\}  \\
p_{i} &=&\min \left\{ \bar{p}_{i},\max \left\{ \left( 1+\gamma \right)
\left( \sum_{j}p_{j}a_{ji}+c_{i}-x_{i}\right) -\gamma \bar{p}_{i},0\right\}
\right\}  \\
\underset{\left\vert D\right\vert \times \left\vert D\right\vert }{A_{D}} &=&%
\underset{\left\vert D\right\vert \times N}{\underbrace{S^{\prime }}}%
\underset{n\times n}{\underbrace{A}}\underset{n\times \left\vert
D\right\vert }{\underbrace{S}} \\
S_{ij} &=&\left\{ 
\begin{array}{ccc}
1 & , & \text{if }j\in D\text{ and }i\leq j \\ 
0 & , & \text{otherwise}%
\end{array}%
\right.  \\
e_{k} &=&\left[ 
\begin{array}{ccccc}
0 & ... & \underset{k^{th}\text{ entry}}{\underbrace{1}} & ... & 0%
\end{array}%
\right]  \\
I_{D} &=&\left\vert D\right\vert \times \left\vert D\right\vert \text{
identity matrix} \\
\tilde{\theta} &=&\left( A,b,c,w,\bar{p},\delta ,\gamma \right) 
\end{eqnarray*}%
Let's write it in vector/matrix form%
\[
L\left( \tilde{\theta}\right) =E\left\{ \underset{1\times 1}{\underbrace{%
\underbrace{\min \left( \underset{1\times n}{\underbrace{x}},\underset{%
1\times n}{\underbrace{w}}\right) }\underset{n\times 1}{\underbrace{1_{n}}}}}%
+\left( 1+\gamma \right) \left[ \underset{1\times 1}{\underbrace{\underset{%
1\times n}{\underbrace{\left( x-w\right) }}\underset{n\times \left\vert
D\right\vert }{\underbrace{S}}\underset{\left\vert D\right\vert \times
\left\vert D\right\vert }{\underbrace{\left( \underset{\left\vert
D\right\vert \times \left\vert D\right\vert }{\underbrace{I_{D}}}-\left(
1+\gamma \right) \underset{\left\vert D\right\vert \times n}{\underbrace{%
S^{T}}}\underset{n\times n}{\underbrace{A}}\underset{n\times \left\vert
D\right\vert }{\underbrace{S}}\right) ^{-1}}}\underset{\left\vert
D\right\vert \times 1}{\underbrace{1_{D}}}}}\right] \right\} 
\]%
\begin{eqnarray*}
&&x\text{ drawn from joint cdf }F\left( x;\tilde{\theta}\right) \text{ and
joint pdf }f\left( x;\tilde{\theta}\right)  \\
D &=&\left\{ i:p_{i}<\bar{p}_{i}\right\}  \\
\left\vert D\right\vert  &=&\#\text{ elements in }D \\
p &=&\min \left\{ \bar{p},\max \left\{ \left( 1+\gamma \right) \left( 
\underset{1\times n}{\underbrace{\underset{1\times n}{\underbrace{p}}%
\underset{n\times n}{\underbrace{A}}}}\underset{1\times n}{+\underbrace{c-x}}%
\right) -\gamma \bar{p},0\right\} \right\}  \\
S_{ij} &=&\left\{ 
\begin{array}{ccc}
1 & , & \text{if }j\in D\text{ and }i\leq j \\ 
0 & , & \text{otherwise}%
\end{array}%
\right.  \\
I_{D} &=&\left\vert D\right\vert \times \left\vert D\right\vert \text{
identity matrix} \\
1_{D} &=&\underset{\left\vert D\right\vert \times 1}{\left[ 
\begin{array}{ccccc}
1 & ... & 1 & ... & 1%
\end{array}%
\right] }^{T} \\
\tilde{\theta} &=&\left( A,b,c,w,\bar{p},\delta ,\gamma \right) 
\end{eqnarray*}%
\begin{eqnarray*}
p_{i} &<&\bar{p}_{i} \\
p_{i} &<&\bar{p}_{i}
\end{eqnarray*}%
\begin{eqnarray*}
case1 &:&\bar{p}\leq \max \left\{ \left( 1+\gamma \right) \left(
pA+c-x\right) -\gamma \bar{p},0\right\}  \\
case1a &:&\left( 1+\gamma \right) \left( pA+c-x\right) -\gamma \bar{p}>0 \\
case1b &:&\left( 1+\gamma \right) \left( pA+c-x\right) -\gamma \bar{p}\leq 0
\end{eqnarray*}%
\begin{eqnarray*}
case2 &:&\bar{p}>\max \left\{ \left( 1+\gamma \right) \left( pA+c-x\right)
-\gamma \bar{p},0\right\}  \\
case2a &:&\left( 1+\gamma \right) \left( pA+c-x\right) -\gamma \bar{p}>0 \\
case2b &:&\left( 1+\gamma \right) \left( pA+c-x\right) -\gamma \bar{p}\leq 0
\end{eqnarray*}%
\begin{eqnarray*}
&&A\text{ is }n\times n \\
&&b,c,w,\bar{p},\delta ,x\text{ are }1\times n \\
&&\gamma \text{ is }1\times 1 \\
&&F\left( x;\tilde{\theta}\right) \text{ is }1\times 1 \\
&&vec\left( \tilde{\theta}\right) \text{ is }\left( n^{2}+5n+1\right) \times
1
\end{eqnarray*}

We solve%
\begin{eqnarray*}
&&\max_{\theta }L\left( x;\tilde{\theta}\right)  \\
&&s.t. \\
&&...
\end{eqnarray*}%
where $\theta $ is a subset of $\tilde{\theta}$. 

To compute numerically, we want to compute%
\[
\nabla _{\theta }L\left( x;\tilde{\theta}\right) 
\]%
Let's make a change of variables so that we can draw the random variables
from the uniform distribution $U$ and then transform them%
\[
x=F^{-1}\left( u;\tilde{\theta}\right) 
\]%
\begin{eqnarray*}
L\left( \tilde{\theta}\right)  &=&\int_{\left[ 0,1\right] ^{n}}\left\{ \min
\left( x,w\right) 1_{n}+\left( 1+\gamma \right) \left[ \left( x-w\right)
S\left( I_{D}-\left( 1+\gamma \right) S^{T}AS\right) ^{-1}1_{D}\right]
\right\} f\left( x;\tilde{\theta}\right) dx \\
&=&\int_{\left[ 0,1\right] ^{n}}\left\{ \min \left( F^{-1}\left( u;\tilde{%
\theta}\right) ,w\right) 1_{n}+\left( 1+\gamma \right) \left[ \left(
F^{-1}\left( u;\tilde{\theta}\right) -w\right) S\left( I_{D}-\left( 1+\gamma
\right) S^{T}AS\right) ^{-1}1_{D}\right] \right\} c\left( u;\tilde{\theta}%
\right) du
\end{eqnarray*}%
where $c\left( u;\tilde{\theta}\right) =1$ for iid uniform. Now let's
consider the beta distribution with parameters $\alpha $ and $\beta $ such
that $P\left( c_{i}X_{i}>w_{i}\right) =\delta _{i}$. Then we have%
\begin{eqnarray*}
F\left( x;\tilde{\theta}\right)  &=&\frac{1}{B\left( \alpha ,\beta \right) }%
x^{\alpha -1}\left( 1-x\right) ^{\beta -1} \\
&=&\frac{1}{B\left( 1,\beta \right) }\left( 1-x\right) ^{\beta -1} \\
&=&1-\left( 1-x\right) ^{\beta }
\end{eqnarray*}%
Therefore%
\begin{eqnarray*}
\delta _{i} &=&P\left( c_{i}X_{i}>w_{i}\right)  \\
&=&1-P\left( c_{i}X_{i}\leq w_{i}\right)  \\
&=&1-F_{i}\left( \frac{w_{i}}{c_{i}};\tilde{\theta}\right)  \\
&=&1-\left( 1-\left( 1-\frac{w_{i}}{c_{i}}\right) ^{\beta _{i}}\right)  \\
&=&\left( 1-\frac{w_{i}}{c_{i}}\right) ^{\beta _{i}}
\end{eqnarray*}%
We then deduce that 
\[
\beta _{i}=\frac{\log \delta _{i}}{\log \left( 1-\frac{w_{i}}{c_{i}}\right) }
\]%
The marginal cdf are%
\[
F\left( x;\tilde{\theta}\right) =1-\left( 1-x\right) ^{\frac{\log \delta _{i}%
}{\log \left( 1-\frac{w_{i}}{c_{i}}\right) }}
\]%
The inverse marginal cdf is%
\[
F^{-1}\left( u_{i};\tilde{\theta}\right) =1-\left( 1-u_{i}\right) ^{\frac{%
\log \left( 1-\frac{w_{i}}{c_{i}}\right) }{\log \delta _{i}}}
\]%
Taking derivatives, we have%
\begin{eqnarray*}
\frac{\partial F^{-1}\left( u_{i};\tilde{\theta}\right) }{\partial c_{i}}
&=&-\left( 1-u_{i}\right) ^{\frac{\log \left( 1-\frac{w_{i}}{c_{i}}\right) }{%
\log \delta _{i}}}\frac{w_{i}\log \left( 1-u_{i}\right) }{c_{i}^{2}\log
\delta _{i}}\left( 1-\frac{w_{i}}{c_{i}}\right) ^{-1} \\
\frac{\partial F^{-1}\left( u_{i};\tilde{\theta}\right) }{\partial c_{j}}
&=&0
\end{eqnarray*}%
Now for the whole loss%
\begin{eqnarray*}
&&\frac{\partial }{\partial c_{i}}E\left[ \sum_{k=1}^{n}\min \left(
c_{k}F_{k}^{-1}\left( u_{k};c_{k}\right) ,w_{k}\right) +\sum_{k=1}^{n}\left(
1+\gamma \right) \left( c_{k}F_{k}^{-1}\left( u_{k};c_{k}\right)
-w_{k}\right) u_{k}\left( x\right) \right]  \\
&&E\left[ \sum_{k=1}^{n}\frac{\partial }{\partial c_{i}}\left\{ \min \left(
c_{k}F_{k}^{-1}\left( u_{k};c_{k}\right) ,w_{k}\right) \right\}
+\sum_{k=1}^{n}\left( 1+\gamma \right) \frac{\partial }{\partial c_{i}}%
\left\{ \left( c_{k}F_{k}^{-1}\left( u_{k};c_{k}\right) -w_{k}\right)
u_{k}\left( x\right) \right\} \right]  \\
&&E\left[ \sum_{k=1}^{n}\frac{\partial }{\partial c_{i}}\left\{ \min \left(
c_{k}F_{k}^{-1}\left( u_{k};c_{k}\right) ,w_{k}\right) \right\}
+\sum_{k=1}^{n}\left( 1+\gamma \right) \left[ \frac{\partial }{\partial c_{i}%
}\left\{ \left( c_{k}F_{k}^{-1}\left( u_{k};c_{k}\right) -w_{k}\right)
\right\} u_{k}\left( x\right) +\left( c_{k}F_{k}^{-1}\left(
u_{k};c_{k}\right) -w_{k}\right) \frac{\partial }{\partial c_{i}}\left\{
u_{k}\left( x\right) \right\} \right] \right]  \\
&&E\left[ \sum_{k=1}^{n}\frac{\partial \left[ c_{k}F_{k}^{-1}\left(
u_{k};c_{k}\right) \right] }{\partial c_{i}}1\left\{ c_{k}F_{k}^{-1}\left(
u_{k};c_{k}\right) <w_{k}\right\} +\left( 1+\gamma \right) \sum_{k=1}^{n}%
\frac{\partial \left[ c_{k}F_{k}^{-1}\left( u_{k};c_{k}\right) \right] }{%
\partial c_{i}}u_{k}\left( x\right) \right]  \\
&&E\left[ \sum_{k=1}^{n}\left\{ 1\left\{ c_{k}x_{k}<w_{k}\right\} +\left(
1+\gamma \right) u_{k}\left( x\right) \right\} \left( c_{k}\frac{\partial
F_{k}^{-1}\left( u_{k};c_{k}\right) }{\partial c_{i}}+F_{k}^{-1}\left(
u_{k};c_{k}\right) \frac{\partial c_{k}}{\partial c_{i}}\right) \right]  \\
&&E\left[ \sum_{k=1}^{n}\left\{ 1\left\{ c_{k}x_{k}<w_{k}\right\} +\left(
1+\gamma \right) u_{k}\left( x\right) \right\} \left( c_{k}\frac{\partial
F_{k}^{-1}\left( u_{k};c_{k}\right) }{\partial c_{k}}+x_{k}\right) \right] 
\end{eqnarray*}%
\begin{eqnarray*}
\frac{\partial }{\partial c_{i}}\left\{ \left( c_{k}F_{k}^{-1}\left(
u_{k};c_{k}\right) -w_{k}\right) \right\}  &=&\frac{\partial }{\partial c_{i}%
}c_{k}F_{k}^{-1}\left( u_{k};c_{k}\right)  \\
&=&c_{k}\frac{\partial F_{k}^{-1}\left( u_{k};c_{k}\right) }{\partial c_{i}}%
+x_{k}\frac{\partial c_{k}}{\partial c_{i}} \\
&=&%
\begin{array}{ccc}
-\left( 1-u_{i}\right) ^{\frac{\log \left( 1-\frac{w_{i}}{c_{i}}\right) }{%
\log \delta _{i}}}\frac{w_{i}\log \left( 1-u_{i}\right) }{c_{i}\log \delta
_{i}}\left( 1-\frac{w_{i}}{c_{i}}\right) ^{-1}+F_{i}^{-1}\left(
u_{i};c_{i}\right)  & , & i=k \\ 
0 & , & \text{otherwise}%
\end{array}
\\
\frac{\partial }{\partial c_{i}}\left\{ \min \left( c_{k}F_{k}^{-1}\left(
u_{k};c_{k}\right) ,w_{k}\right) \right\}  &=&%
\begin{array}{ccc}
F_{k}^{-1}\left( u_{k};c_{k}\right) +c_{k}\frac{\partial }{\partial c_{i}}%
F_{k}^{-1}\left( u_{k};c_{k}\right)  & , & c_{k}F_{k}^{-1}\left(
u_{k};c_{k}\right) <w_{k} \\ 
0 & , & \text{otherwise}%
\end{array}
\\
&=&%
\begin{array}{ccc}
x_{k}+c_{k}\frac{\partial }{\partial c_{i}}F_{k}^{-1}\left(
u_{k};c_{k}\right)  & , & c_{k}x_{k}<w_{k} \\ 
0 & , & \text{otherwise}%
\end{array}
\\
\frac{\partial u_{k}\left( c\circ x\right) }{\partial c_{j}} &=&0
\end{eqnarray*}%
\begin{eqnarray*}
\frac{\partial \left[ \left( c_{k}x_{k}-w_{k}\right) u_{k}\left( c\circ
x\right) \right] }{\partial c_{j}} &=&u_{k}\left( c\circ x\right) \frac{%
\partial \left( c_{k}x_{k}-w_{k}\right) }{\partial c_{j}}+\left(
c_{k}x_{k}-w_{k}\right) \frac{\partial u_{k}\left( c\circ x\right) }{%
\partial c_{j}} \\
\frac{\partial \left( c_{k}x_{k}-w_{k}\right) }{\partial c_{j}} &=&x_{k}%
\frac{\partial c_{k}}{\partial c_{j}}=\left\{ 
\begin{array}{ccc}
x_{k} & , & j=k \\ 
0 & , & otherwise%
\end{array}%
\right.  \\
\frac{\partial u_{k}\left( c\circ x\right) }{\partial c_{j}} &=&0
\end{eqnarray*}%
\begin{eqnarray*}
\left( \nabla _{c}L\right) _{i} &=&\left\{ x_{i}+c_{i}\frac{\partial
F_{i}^{-1}\left( u_{i};c_{i}\right) }{\partial c_{i}}\right\} 1\left\{
c_{i}x_{i}<w_{i}\right\} +\left\{ x_{i}+c_{i}\frac{\partial F_{k}^{-1}\left(
u_{i};c_{i}\right) }{\partial c_{i}}\right\}  \\
&=&\left( 1\left\{ c_{i}x_{i}<w_{i}\right\} +1\right) \left\{ x_{i}+c_{i}%
\frac{\partial F_{k}^{-1}\left( u_{i};c_{i}\right) }{\partial c_{i}}\right\} 
\end{eqnarray*}%
\begin{eqnarray*}
\frac{\partial \left[ e_{k}\left( I_{D}-\left( 1+\gamma \right) A_{D}\right)
^{-1}1_{D}\right] }{\partial A_{D}} &=&\left( 1+\gamma \right) \left(
I_{D}-\left( 1+\gamma \right) A_{D}\right) ^{-T}e_{k}^{T}1_{D}^{T}\left(
I_{D}-\left( 1+\gamma \right) A_{D}\right) ^{-T} \\
&=&\left( 1+\gamma \right) \left[ e_{k}\left( I_{D}-\left( 1+\gamma \right)
A_{D}\right) ^{-1}\right] ^{T}\left[ \left( I_{D}-\left( 1+\gamma \right)
A_{D}\right) ^{-1}1_{D}\right] ^{T} \\
&=&\left( 1+\gamma \right) \left( I_{D}-\left( 1+\gamma \right) A_{D}\right)
^{-T}e_{k}^{T}u_{D}^{T}
\end{eqnarray*}

\section{Old}

We would like to compute the gradient of expected losses%
\[
\nabla _{A}\left\{ E\left[ \sum_{k=1}^{n}\min \left( x_{k},w_{k}\right)
+\sum_{k=1}^{n}\left( x_{k}-w_{k}\right) u_{k}\left( x\right) \right]
\right\} 
\]%
where $\nabla _{A}$ means the gradient with respect to the entries of the
matrix $A$. Since $x_{k}$ and $w_{k}$ do not depend on $A$, and expectations
are linear,%
\[
\nabla _{A}\left\{ E\left[ \sum_{k=1}^{n}\min \left( x_{k},w_{k}\right)
+\sum_{k=1}^{n}\left( x_{k}-w_{k}\right) u_{k}\left( x\right) \right]
\right\} =E\sum_{k=1}^{n}\left( x_{k}-w_{k}\right) \nabla _{A}u_{k}\left(
x\right) 
\]%
To compute $\nabla _{A}u_{k}\left( x\right) $, we use that%
\[
u_{k}\left( x\right) =\left\{ 
\begin{array}{ccc}
0 & , & \text{if }k\notin D\left( A\right)  \\ 
e_{k}\left( I_{D}-\left( 1+\gamma \right) A_{D}\right) ^{-1}1_{D} & , & 
\text{if }k\in D\left( A\right) 
\end{array}%
\right. 
\]%
where $D$ is the set of defaulting nodes, $I_{D}$ is the $\left\vert
D\right\vert \times \left\vert D\right\vert $ identity matrix, $e_{k}$ is a
row vector with a $1$ in column $k$ and zeros otherwise, and $A_{D}$ is the
matrix $A$ restricted to $D$. We can write%
\[
A_{D}=S^{\prime }AS
\]%
for a matrix $S$ with dimension $n\times \left\vert D\right\vert $ 
\[
S_{ij}=\left\{ 
\begin{array}{ccc}
1 & , & \text{if }j\in D\text{ and }i\leq j \\ 
0 & , & \text{otherwise}%
\end{array}%
\right. 
\]%
so that $A_{D}$ has dimension $\left\vert D\right\vert \times \left\vert
D\right\vert $. For example, if $n=3$ and $D=\left\{ 1,3\right\} $ 
\[
S=\left[ 
\begin{array}{cc}
1 & 0 \\ 
0 & 0 \\ 
0 & 1%
\end{array}%
\right] 
\]%
and%
\begin{eqnarray*}
A_{D} &=&\left[ 
\begin{array}{ccc}
1 & 0 & 0 \\ 
0 & 0 & 1%
\end{array}%
\right] \left[ 
\begin{array}{ccc}
a_{11} & a_{12} & a_{13} \\ 
a_{21} & a_{22} & a_{23} \\ 
a_{31} & a_{32} & a_{33}%
\end{array}%
\right] \left[ 
\begin{array}{cc}
1 & 0 \\ 
0 & 0 \\ 
0 & 1%
\end{array}%
\right]  \\
&=&\left[ 
\begin{array}{cc}
a_{11} & a_{13} \\ 
a_{31} & a_{33}%
\end{array}%
\right] 
\end{eqnarray*}%
We also note that $D$ depends on $A$ 
\begin{eqnarray*}
D\left( A\right)  &=&\left\{ i:p_{i}\left( A\right) <\bar{p}_{i}\right\}  \\
&=&\left\{ i:\min \left\{ \bar{p}_{i},\max \left\{ \sum_{j}p_{j}\left(
A\right) a_{ji}+c_{i}-x_{i},0\right\} \right\} <\bar{p}_{i}\right\} 
\end{eqnarray*}%
However, $p_{i}\left( A\right) $ is continuous in $A$, so small changes in $A
$ can only affect $D$ if%
\[
p_{i}\left( A\right) =\bar{p}_{i}
\]%
We now compute%
\[
\nabla _{A}u_{k}\left( x\right) =\left\{ 
\begin{array}{ccc}
0 & , & \text{if }k\notin D\left( A\right)  \\ 
\nabla _{A}e_{k}\left( I_{D}-S^{T}AS\right) ^{-1}1_{D} & , & \text{if }k\in
D\left( A\right) 
\end{array}%
\right. 
\]%
Use the formulas%
\begin{eqnarray*}
\frac{\partial a^{T}X^{-1}b}{\partial X} &=&-X^{-T}ab^{T}X^{-T} \\
\partial \left( XY\right)  &=&\left( \partial X\right) Y+X\partial Y \\
\partial \left( S^{T}AS\right)  &=&\partial \left( S^{T}A\right) S+\left(
S^{T}A\right) \partial S \\
&=&\left[ \partial \left( S^{T}\right) A+S^{T}\partial A\right] S+\left(
S^{T}A\right) \partial S \\
&=&S^{T}\left( \partial A\right) S
\end{eqnarray*}%
to get%
\begin{eqnarray*}
\frac{\partial \left[ e_{k}\left( I_{D}-\left( 1+\gamma \right) A_{D}\right)
^{-1}1_{D}\right] }{\partial A_{D}} &=&\left( 1+\gamma \right) \left(
I_{D}-\left( 1+\gamma \right) A_{D}\right) ^{-T}e_{k}^{T}1_{D}^{T}\left(
I_{D}-\left( 1+\gamma \right) A_{D}\right) ^{-T} \\
&=&\left( 1+\gamma \right) \left[ e_{k}\left( I_{D}-\left( 1+\gamma \right)
A_{D}\right) ^{-1}\right] ^{T}\left[ \left( I_{D}-\left( 1+\gamma \right)
A_{D}\right) ^{-1}1_{D}\right] ^{T} \\
&=&\left( 1+\gamma \right) \left( I_{D}-\left( 1+\gamma \right) A_{D}\right)
^{-T}e_{k}^{T}u_{D}^{T}
\end{eqnarray*}%
and 
\[
\frac{\partial \left[ e_{k}\left( I_{D}-A_{D}\right) ^{-1}1_{D}\right] }{%
\partial A_{D^{c}}}=0
\]%
where $D^{c}$ is the complement of $D$.%
\[
\nabla _{A}u_{k}\left( x\right) =S\left[ \nabla _{A_{D}}u_{k}\left( x\right) %
\right] S^{T}
\]%
Finally, 
\[
E\sum_{k=1}^{n}\left( x_{k}-w_{k}\right) S\left[ \nabla _{A_{D}}u_{k}\left(
x\right) \right] S^{T}
\]%
Now we compute the gradient with respect to $c$, assuming that $x_{k}$ is a
random variable between $0$ and $1$ and thus total losses are%
\[
E\left[ \sum_{k=1}^{n}\min \left( c_{k}x_{k},w_{k}\right)
+\sum_{k=1}^{n}\left( c_{k}x_{k}-w_{k}\right) u_{k}\left( c\circ x\right) %
\right] 
\]%
where $c\circ x$ denotes element-wise multiplication.

We compute%
\[
\nabla _{c}\left\{ E\left[ \sum_{k=1}^{n}\min \left( c_{k}x_{k},w_{k}\right)
+\sum_{k=1}^{n}\left( c_{k}x_{k}-w_{k}\right) u_{k}\left( c\circ x\right) %
\right] \right\} 
\]%
Using%
\begin{eqnarray*}
\frac{\partial \min \left( c_{k}x_{k},w_{k}\right) }{\partial c_{j}}
&=&\left\{ 
\begin{array}{ccc}
x_{k} & , & c_{k}<\frac{w_{k}}{x_{k}}\text{ and }j=k \\ 
0 & , & otherwise%
\end{array}%
\right.  \\
\nabla _{c}\left\{ \min \left( c_{k}x_{k},w_{k}\right) \right\}  &=&\left( 
\begin{array}{ccccc}
0 & \cdots  & 1_{\left\{ c_{k}<\frac{w_{k}}{x_{k}}\right\} }x_{k} & \cdots 
& 0%
\end{array}%
\right)  \\
\sum_{k=1}^{n}\nabla _{c}\left\{ \min \left( c_{k}x_{k},w_{k}\right)
\right\}  &=&\left( 
\begin{array}{ccccc}
1_{\left\{ c_{1}<\frac{w_{1}}{x_{1}}\right\} }x_{1} & \cdots  & 1_{\left\{
c_{k}<\frac{w_{k}}{x_{k}}\right\} }x_{k} & \cdots  & 1_{\left\{ c_{n}<\frac{%
w_{n}}{x_{n}}\right\} }x_{n}%
\end{array}%
\right) 
\end{eqnarray*}%
and%
\begin{eqnarray*}
\frac{\partial \left[ \left( c_{k}x_{k}-w_{k}\right) u_{k}\left( c\circ
x\right) \right] }{\partial c_{j}} &=&u_{k}\left( c\circ x\right) \frac{%
\partial \left( c_{k}x_{k}-w_{k}\right) }{\partial c_{j}}+\left(
c_{k}x_{k}-w_{k}\right) \frac{\partial u_{k}\left( c\circ x\right) }{%
\partial c_{j}} \\
\frac{\partial \left( c_{k}x_{k}-w_{k}\right) }{\partial c_{j}} &=&x_{k}%
\frac{\partial c_{k}}{\partial c_{j}}=\left\{ 
\begin{array}{ccc}
x_{k} & , & j=k \\ 
0 & , & otherwise%
\end{array}%
\right.  \\
\frac{\partial u_{k}\left( c\circ x\right) }{\partial c_{j}} &=&0
\end{eqnarray*}%
we get%
\begin{eqnarray*}
&&\nabla _{c}\left\{ E\left[ \sum_{k=1}^{n}\min \left(
c_{k}x_{k},w_{k}\right) +\sum_{k=1}^{n}\left( c_{k}x_{k}-w_{k}\right)
u_{k}\left( c\circ x\right) \right] \right\}  \\
&=&E\left[ \sum_{k=1}^{n}\nabla _{c}\min \left( c_{k}x_{k},w_{k}\right)
+\sum_{k=1}^{n}\nabla _{c}\left[ \left( c_{k}x_{k}-w_{k}\right) u_{k}\left(
c\circ x\right) \right] \right]  \\
&=&E\left[ 
\begin{array}{c}
\left( 
\begin{array}{ccccc}
1_{\left\{ c_{1}<\frac{w_{1}}{x_{1}}\right\} }x_{1} & \cdots  & 1_{\left\{
c_{k}<\frac{w_{k}}{x_{k}}\right\} }x_{k} & \cdots  & 1_{\left\{ c_{n}<\frac{%
w_{n}}{x_{n}}\right\} }x_{n}%
\end{array}%
\right)  \\ 
+\left( 
\begin{array}{ccccc}
x_{1}u_{1}\left( c\circ x\right)  & \cdots  & x_{k}u_{k}\left( c\circ
x\right)  & \cdots  & x_{n}u_{n}\left( c\circ x\right) 
\end{array}%
\right) 
\end{array}%
\right]  \\
&=&E\left[ \left( 
\begin{array}{ccccc}
\left[ 1_{\left\{ c_{1}<\frac{w_{1}}{x_{1}}\right\} }+u_{1}\left( c\circ
x\right) \right] x_{1} & \cdots  & \left[ 1_{\left\{ c_{k}<\frac{w_{k}}{x_{k}%
}\right\} }+u_{k}\left( c\circ x\right) \right] x_{k} & \cdots  & \left[
1_{\left\{ c_{n}<\frac{w_{n}}{x_{n}}\right\} }+u_{n}\left( c\circ x\right) %
\right] x_{n}%
\end{array}%
\right) \right] 
\end{eqnarray*}%
We pick the parameter $\beta $ of the beta distribution with pdf%
\[
f\left( x\right) =\frac{1}{B\left( \alpha ,\beta \right) }x^{\alpha
-1}x^{\beta -1}
\]%
and cdf%
\[
F\left( x\right) =\frac{1}{B\left( \alpha ,\beta \right) }%
\int_{0}^{x}x^{\alpha -1}x^{\beta -1}dx
\]%
so that

\begin{eqnarray*}
\delta  &=&P\left( cx>w\right)  \\
&=&1-F\left( \frac{w}{c},\alpha ,\beta \right)  \\
&=&1-\frac{1}{B\left( \alpha ,\beta \right) }\int_{0}^{\frac{w}{c}}x^{\alpha
-1}x^{\beta -1}dx
\end{eqnarray*}%
Solving for $\beta $ gives a solution%
\[
\beta =\beta \left( c,w,\delta ,\alpha \right) 
\]%
The CDF and PDF\ are then%
\begin{eqnarray*}
F\left( x;c\right)  &=&\frac{1}{B\left( \alpha ,\beta \left( c,w,\delta
,\alpha \right) \right) }\int_{0}^{x}x^{\alpha -1}x^{\beta \left( c,w,\delta
,\alpha \right) -1}dx \\
f\left( x;c\right)  &=&\frac{1}{B\left( \alpha ,\beta \left( c,w,\delta
,\alpha \right) \right) }x^{\alpha -1}x^{\beta \left( c,w,\delta ,\alpha
\right) -1}
\end{eqnarray*}%
To compute 
\[
\nabla _{c}\left\{ E\left[ \sum_{k=1}^{n}\min \left( c_{k}x_{k},w_{k}\right)
+\sum_{k=1}^{n}\left( c_{k}x_{k}-w_{k}\right) u_{k}\left( c\circ x\right) %
\right] \right\} 
\]%
we then note that%
\begin{eqnarray*}
&&\nabla _{c}\left\{ E\left[ \sum_{k=1}^{n}\min \left(
c_{k}x_{k},w_{k}\right) +\sum_{k=1}^{n}\left( c_{k}x_{k}-w_{k}\right)
u_{k}\left( c\circ x\right) \right] \right\}  \\
&=&\nabla _{c}\int_{0}^{1}...\int_{0}^{1}\left[ \sum_{k=1}^{n}\min \left(
c_{k}x_{k},w_{k}\right) +\sum_{k=1}^{n}\left( c_{k}x_{k}-w_{k}\right)
u_{k}\left( c\circ x\right) \right] f\left( x;c\right) dx_{1}...dx_{n} \\
&=&\int_{0}^{1}...\int_{0}^{1}\nabla _{c}\left\{ \left[ \sum_{k=1}^{n}\min
\left( c_{k}x_{k},w_{k}\right) +\sum_{k=1}^{n}\left( c_{k}x_{k}-w_{k}\right)
u_{k}\left( c\circ x\right) \right] f\left( x;c\right) \right\}
dx_{1}...dx_{n} \\
&=&\int_{0}^{1}...\int_{0}^{1}\nabla _{c}\left[ \sum_{k=1}^{n}\min \left(
c_{k}x_{k},w_{k}\right) +\sum_{k=1}^{n}\left( c_{k}x_{k}-w_{k}\right)
u_{k}\left( c\circ x\right) \right] f\left( x;c\right) dx_{1}...dx_{n} \\
&&+\int_{0}^{1}...\int_{0}^{1}\left[ \sum_{k=1}^{n}\min \left(
c_{k}x_{k},w_{k}\right) +\sum_{k=1}^{n}\left( c_{k}x_{k}-w_{k}\right)
u_{k}\left( c\circ x\right) \right] \nabla _{c}f\left( x;c\right)
dx_{1}...dx_{n} \\
&=&\int_{0}^{1}...\int_{0}^{1}\nabla _{c}\left[ \sum_{k=1}^{n}\min \left(
c_{k}x_{k},w_{k}\right) +\sum_{k=1}^{n}\left( c_{k}x_{k}-w_{k}\right)
u_{k}\left( c\circ x\right) \right] f\left( x;c\right) dx_{1}...dx_{n} \\
&&+\int_{0}^{1}...\int_{0}^{1}L\left( x\right) \frac{\nabla _{c}f\left(
x;c\right) }{f\left( x;c\right) }f\left( x;c\right) dx_{1}...dx_{n} \\
&=&E\left[ \nabla _{c}\left[ \sum_{k=1}^{n}\min \left(
c_{k}x_{k},w_{k}\right) +\sum_{k=1}^{n}\left( c_{k}x_{k}-w_{k}\right)
u_{k}\left( c\circ x\right) \right] \right] +E\left[ L\left( x\right) \frac{%
\nabla _{c}f\left( x;c\right) }{f\left( x;c\right) }\right] 
\end{eqnarray*}%
For independent random variables $x$%
\[
f\left( x;c\right) =\dprod\limits_{k=1}^{n}f\left( x_{k};c_{k}\right) 
\]%
so that%
\[
\frac{\nabla _{c}f\left( x;c\right) }{f\left( x;c\right) }=\left( \frac{1}{%
f\left( x_{1};c_{1}\right) }\frac{df\left( x_{1};c_{1}\right) }{dc_{1}},..,%
\frac{1}{f\left( x_{k};c_{k}\right) }\frac{df\left( x_{k};c_{k}\right) }{%
dc_{k}},..,\frac{1}{f\left( x_{n};c_{n}\right) }\frac{df\left(
x_{n};c_{n}\right) }{dc_{n}}\right) 
\]%
Assume $\alpha =1$. Then%
\begin{eqnarray*}
\delta  &=&1-\frac{1}{B\left( 1,\beta \right) }\int_{0}^{\frac{w}{c}%
}x^{\beta -1}dx \\
\delta  &=&\left( 1-\frac{w}{c}\right) ^{\beta } \\
\beta  &=&\frac{\log \delta }{\log \left( 1-\frac{w}{c}\right) }
\end{eqnarray*}%
and%
\begin{eqnarray*}
f_{k}\left( x;1,\beta \right)  &=&\beta \left( 1-x\right) ^{\beta -1} \\
f_{k}\left( x;1,\frac{\log \delta }{\log \left( 1-\frac{w}{c}\right) }%
\right)  &=&\frac{\log \delta }{\log \left( 1-\frac{w}{c}\right) }\left(
1-x\right) ^{\frac{\log \delta }{\log \left( 1-\frac{w}{c}\right) }-1} \\
\frac{1}{f\left( x_{k};c_{k}\right) }\frac{df\left( x_{k};c_{k}\right) }{%
dc_{k}} &=&\frac{1}{\frac{\log \delta }{\log \left( 1-\frac{w}{c}\right) }%
\left( 1-x\right) ^{\frac{\log \delta }{\log \left( 1-\frac{w}{c}\right) }-1}%
}\left( \left\{ \frac{\log \delta }{\log \left( 1-\frac{w}{c}\right) }%
\right\} \frac{d}{dc}\left\{ \left( 1-x\right) ^{\frac{\log \delta }{\log
\left( 1-\frac{w}{c}\right) }-1}\right\} +\left\{ \left( 1-x\right) ^{\frac{%
\log \delta }{\log \left( 1-\frac{w}{c}\right) }-1}\right\} \frac{d}{dc}%
\left\{ \frac{\log \delta }{\log \left( 1-\frac{w}{c}\right) }\right\}
\right)  \\
&=&\frac{1}{\left( 1-x\right) ^{\frac{\log \delta }{\log \left( 1-\frac{w}{c}%
\right) }}}\frac{d}{dc}\left\{ \left( 1-x\right) ^{\frac{\log \delta }{\log
\left( 1-\frac{w}{c}\right) }}\right\} -\frac{1}{\log \left( 1-\frac{w}{c}%
\right) }\frac{1}{\left( c-w\right) }\frac{w}{c} \\
&=&-\frac{w\left( \log \left( 1-\frac{w}{c}\right) +\log \left( 1-x\right)
\log \left( \delta \right) \right) }{c\left( c-w\right) \log \left( 1-\frac{w%
}{c}\right) ^{2}}
\end{eqnarray*}

If we use a gaussian copula with correlation matrix $R$ and marginals 
\[
f\left( x_{k};c_{k}\right) =\frac{1}{B\left( \alpha _{k},\beta \left(
c_{k},w_{k},\delta _{k},\alpha _{k}\right) \right) }x^{\alpha
_{k}-1}x^{\beta \left( c_{k},w_{k},\delta _{k},\alpha _{k}\right) -1}
\]%
the density of the copula is%
\[
c_{R}^{Gauss}\left( u\right) =\frac{1}{\sqrt{\det R}}\exp \left( -\frac{1}{2}%
\left( 
\begin{array}{c}
\Phi ^{-1}\left( u_{1}\right)  \\ 
... \\ 
\Phi ^{-1}\left( u_{n}\right) 
\end{array}%
\right) ^{T}\left( R^{-1}-I\right) \left( 
\begin{array}{c}
\Phi ^{-1}\left( u_{1}\right)  \\ 
... \\ 
\Phi ^{-1}\left( u_{n}\right) 
\end{array}%
\right) \right) 
\]%
Thus,%
\[
f\left( x;c\right) =c_{R}^{Gauss}\left( F\left( x_{1};c_{1}\right)
,...,F\left( x_{n};c_{n}\right) \right) f\left( x_{1};c_{1}\right)
...f\left( x_{n};c_{n}\right) 
\]%
We can now compute the $i^{th}$ component of $\frac{\nabla _{c}f\left(
x;c\right) }{f\left( x;c\right) }:$ 
\[
c_{R,k}^{Gauss}\left( F\left( x_{1};c_{1}\right) ,...,F\left(
x_{n};c_{n}\right) \right) f\left( x_{k};c_{k}\right) \frac{df\left(
x_{k};c_{k}\right) }{dc_{k}}+c_{R}^{Gauss}\left( F\left( x_{1};c_{1}\right)
,...,F\left( x_{n};c_{n}\right) \right) \frac{1}{f\left( x_{k};c_{k}\right) }%
\frac{df\left( x_{k};c_{k}\right) }{dc_{k}}
\]

\bigskip 

\bigskip 

Try%
\begin{eqnarray*}
&&\nabla _{c}\left\{ E\left[ \sum_{k=1}^{n}\min \left( c_{k}F^{-1}\left(
u_{k};c_{k}\right) ,w_{k}\right) +\sum_{k=1}^{n}\left( c_{k}F^{-1}\left(
u_{k};c_{k}\right) -w_{k}\right) u_{k}\left( c\circ F^{-1}\left( x\right)
\right) \right] \right\}  \\
&=&\nabla _{c}\left\{ E\left[ \sum_{k=1}^{n}\min \left(
c_{k}x_{k},w_{k}\right) +\sum_{k=1}^{n}\left( c_{k}x_{k}-w_{k}\right)
u_{k}\left( c\circ x\right) \right] \right\}  \\
&&+\left\{ E\left[ \sum_{k=1}^{n}\min \left( c_{k}\nabla _{c}F^{-1}\left(
u_{k};c_{k}\right) ,w_{k}\right) +\sum_{k=1}^{n}\left( c_{k}\nabla
_{c}F^{-1}\left( u_{k};c_{k}\right) -w_{k}\right) u_{k}\left( c\circ
F^{-1}\left( x\right) \right) \right] \right\} 
\end{eqnarray*}

\end{document}
