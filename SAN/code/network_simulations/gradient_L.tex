%2multibyte Version: 5.50.0.2953 CodePage: 1252

\documentclass{article}
\usepackage{amsmath}

%%%%%%%%%%%%%%%%%%%%%%%%%%%%%%%%%%%%%%%%%%%%%%%%%%%%%%%%%%%%%%%%%%%%%%%%%%%%%%%%%%%%%%%%%%%%%%%%%%%%%%%%%%%%%%%%%%%%%%%%%%%%%%%%%%%%%%%%%%%%%%%%%%%%%%%%%%%%%%%%%%%%%%%%%%%%%%%%%%%%%%%%%%%%%%%%%%%%%%%%%%%%%%%%%%%%%%%%%%%%%%%%%%%%
%TCIDATA{OutputFilter=LATEX.DLL}
%TCIDATA{Version=5.50.0.2953}
%TCIDATA{Codepage=1252}
%TCIDATA{<META NAME="SaveForMode" CONTENT="1">}
%TCIDATA{BibliographyScheme=Manual}
%TCIDATA{Created=Tuesday, September 05, 2017 18:49:10}
%TCIDATA{LastRevised=Tuesday, September 12, 2017 12:28:13}
%TCIDATA{<META NAME="GraphicsSave" CONTENT="32">}
%TCIDATA{<META NAME="DocumentShell" CONTENT="Standard LaTeX\Blank - Standard LaTeX Article">}
%TCIDATA{CSTFile=40 LaTeX article.cst}

\newtheorem{theorem}{Theorem}
\newtheorem{acknowledgement}[theorem]{Acknowledgement}
\newtheorem{algorithm}[theorem]{Algorithm}
\newtheorem{axiom}[theorem]{Axiom}
\newtheorem{case}[theorem]{Case}
\newtheorem{claim}[theorem]{Claim}
\newtheorem{conclusion}[theorem]{Conclusion}
\newtheorem{condition}[theorem]{Condition}
\newtheorem{conjecture}[theorem]{Conjecture}
\newtheorem{corollary}[theorem]{Corollary}
\newtheorem{criterion}[theorem]{Criterion}
\newtheorem{definition}[theorem]{Definition}
\newtheorem{example}[theorem]{Example}
\newtheorem{exercise}[theorem]{Exercise}
\newtheorem{lemma}[theorem]{Lemma}
\newtheorem{notation}[theorem]{Notation}
\newtheorem{problem}[theorem]{Problem}
\newtheorem{proposition}[theorem]{Proposition}
\newtheorem{remark}[theorem]{Remark}
\newtheorem{solution}[theorem]{Solution}
\newtheorem{summary}[theorem]{Summary}
\newenvironment{proof}[1][Proof]{\noindent\textbf{#1.} }{\ \rule{0.5em}{0.5em}}
\input{tcilatex}
\begin{document}


\section{Computing the gradient}

The loss function is given by 
\begin{eqnarray*}
L\left( \tilde{\theta}\right)  &=&E\left[ \sum_{k=1}^{n}\min \left(
x_{k},w_{k}\right) +\sum_{k=1}^{n}\left( 1+\gamma \right) \left(
x_{k}-w_{k}\right) u_{k}\left( x\right) \right]  \\
&&x\text{ drawn from joint cdf }F\left( x;\tilde{\theta}\right)  \\
u_{k}\left( x\right)  &=&\left\{ 
\begin{array}{ccc}
0 & , & \text{if }k\notin D \\ 
e_{k}\left( I_{D}-\left( 1+\gamma \right) A_{D}\right) ^{-1}1_{D} & , & 
\text{if }k\in D%
\end{array}%
\right.  \\
D &=&\left\{ i:p_{i}<\bar{p}_{i}\right\}  \\
p_{i} &=&\min \left\{ \bar{p}_{i},\max \left\{ \left( 1+\gamma \right)
\left( \sum_{j}p_{j}a_{ji}+c_{i}-x_{i}\right) -\gamma \bar{p}_{i},0\right\}
\right\}  \\
\underset{\left\vert D\right\vert \times \left\vert D\right\vert }{A_{D}} &=&%
\underset{\left\vert D\right\vert \times N}{\underbrace{S^{\prime }}}%
\underset{n\times n}{\underbrace{A}}\underset{n\times \left\vert
D\right\vert }{\underbrace{S}} \\
S_{ij} &=&\left\{ 
\begin{array}{ccc}
1 & , & \text{if }j\in D\text{ and }i\leq j \\ 
0 & , & \text{otherwise}%
\end{array}%
\right.  \\
e_{k} &=&\left[ 
\begin{array}{ccccc}
0 & ... & \underset{k^{th}\text{ entry}}{\underbrace{1}} & ... & 0%
\end{array}%
\right]  \\
I_{D} &=&\left\vert D\right\vert \times \left\vert D\right\vert \text{
identity matrix} \\
\tilde{\theta} &=&\left( A,b,c,w,\bar{p},\delta ,\gamma \right) 
\end{eqnarray*}%
Let's write it in vector/matrix form%
\[
L\left( \tilde{\theta}\right) =E\left\{ \underset{1\times 1}{\underbrace{%
\underbrace{\min \left( \underset{1\times n}{\underbrace{x}},\underset{%
1\times n}{\underbrace{w}}\right) }\underset{n\times 1}{\underbrace{1_{n}}}}}%
+\left( 1+\gamma \right) \left[ \underset{1\times 1}{\underbrace{\underset{%
1\times n}{\underbrace{\left( x-w\right) }}\underset{n\times \left\vert
D\right\vert }{\underbrace{S}}\underset{\left\vert D\right\vert \times
\left\vert D\right\vert }{\underbrace{\left( \underset{\left\vert
D\right\vert \times \left\vert D\right\vert }{\underbrace{I_{D}}}-\left(
1+\gamma \right) \underset{\left\vert D\right\vert \times n}{\underbrace{%
S^{T}}}\underset{n\times n}{\underbrace{A}}\underset{n\times \left\vert
D\right\vert }{\underbrace{S}}\right) ^{-1}}}\underset{\left\vert
D\right\vert \times 1}{\underbrace{1_{D}}}}}\right] \right\} 
\]%
\begin{eqnarray*}
&&x\text{ drawn from joint cdf }F\left( x;\tilde{\theta}\right) \text{ and
joint pdf }f\left( x;\tilde{\theta}\right)  \\
D &=&\left\{ i:p_{i}<\bar{p}_{i}\right\}  \\
\left\vert D\right\vert  &=&\#\text{ elements in }D \\
p &=&\min \left\{ \bar{p},\max \left\{ \left( 1+\gamma \right) \left( 
\underset{1\times n}{\underbrace{\underset{1\times n}{\underbrace{p}}%
\underset{n\times n}{\underbrace{A}}}}\underset{1\times n}{+\underbrace{c-x}}%
\right) -\gamma \bar{p},0\right\} \right\}  \\
S_{ij} &=&\left\{ 
\begin{array}{ccc}
1 & , & \text{if }j\in D\text{ and }i\leq j \\ 
0 & , & \text{otherwise}%
\end{array}%
\right.  \\
I_{D} &=&\left\vert D\right\vert \times \left\vert D\right\vert \text{
identity matrix} \\
1_{D} &=&\underset{\left\vert D\right\vert \times 1}{\left[ 
\begin{array}{ccccc}
1 & ... & 1 & ... & 1%
\end{array}%
\right] }^{T} \\
\tilde{\theta} &=&\left( A,b,c,w,\bar{p},\delta ,\gamma \right) 
\end{eqnarray*}%
\begin{eqnarray*}
p_{i} &<&\bar{p}_{i} \\
p_{i} &<&\bar{p}_{i}
\end{eqnarray*}%
\begin{eqnarray*}
case1 &:&\bar{p}\leq \max \left\{ \left( 1+\gamma \right) \left(
pA+c-x\right) -\gamma \bar{p},0\right\}  \\
case1a &:&\left( 1+\gamma \right) \left( pA+c-x\right) -\gamma \bar{p}>0 \\
case1b &:&\left( 1+\gamma \right) \left( pA+c-x\right) -\gamma \bar{p}\leq 0
\end{eqnarray*}%
\begin{eqnarray*}
case2 &:&\bar{p}>\max \left\{ \left( 1+\gamma \right) \left( pA+c-x\right)
-\gamma \bar{p},0\right\}  \\
case2a &:&\left( 1+\gamma \right) \left( pA+c-x\right) -\gamma \bar{p}>0 \\
case2b &:&\left( 1+\gamma \right) \left( pA+c-x\right) -\gamma \bar{p}\leq 0
\end{eqnarray*}%
\begin{eqnarray*}
&&A\text{ is }n\times n \\
&&b,c,w,\bar{p},\delta ,x\text{ are }1\times n \\
&&\gamma \text{ is }1\times 1 \\
&&F\left( x;\tilde{\theta}\right) \text{ is }1\times 1 \\
&&vec\left( \tilde{\theta}\right) \text{ is }\left( n^{2}+5n+1\right) \times
1
\end{eqnarray*}

We solve%
\begin{eqnarray*}
&&\max_{\theta }L\left( x;\tilde{\theta}\right)  \\
&&s.t. \\
&&...
\end{eqnarray*}%
where $\theta $ is a subset of $\tilde{\theta}$. 

To compute numerically, we want to compute%
\[
\nabla _{\theta }L\left( x;\tilde{\theta}\right) 
\]%
Let's make a change of variables so that we can draw the random variables
from the uniform distribution $U$ and then transform them%
\[
x=F^{-1}\left( u;\tilde{\theta}\right) 
\]%
\begin{eqnarray*}
L\left( \tilde{\theta}\right)  &=&\int_{\left[ 0,1\right] ^{n}}\left\{ \min
\left( x,w\right) 1_{n}+\left( 1+\gamma \right) \left[ \left( x-w\right)
S\left( I_{D}-\left( 1+\gamma \right) S^{T}AS\right) ^{-1}1_{D}\right]
\right\} f\left( x;\tilde{\theta}\right) dx \\
&=&\int_{\left[ 0,1\right] ^{n}}\left\{ \min \left( F^{-1}\left( u;\tilde{%
\theta}\right) ,w\right) 1_{n}+\left( 1+\gamma \right) \left[ \left(
F^{-1}\left( u;\tilde{\theta}\right) -w\right) S\left( I_{D}-\left( 1+\gamma
\right) S^{T}AS\right) ^{-1}1_{D}\right] \right\} c\left( u;\tilde{\theta}%
\right) du
\end{eqnarray*}%
where $c\left( u;\tilde{\theta}\right) =1$ for iid uniform. Now let's
consider the beta distribution with parameters $\alpha $ and $\beta $ such
that $P\left( c_{i}X_{i}>w_{i}\right) =\delta _{i}$. Then we have%
\begin{eqnarray*}
F\left( x;\tilde{\theta}\right)  &=&\frac{1}{B\left( \alpha ,\beta \right) }%
x^{\alpha -1}\left( 1-x\right) ^{\beta -1} \\
&=&\frac{1}{B\left( 1,\beta \right) }\left( 1-x\right) ^{\beta -1} \\
&=&1-\left( 1-x\right) ^{\beta }
\end{eqnarray*}%
Therefore%
\begin{eqnarray*}
\delta _{i} &=&P\left( c_{i}X_{i}>w_{i}\right)  \\
&=&1-P\left( c_{i}X_{i}\leq w_{i}\right)  \\
&=&1-F_{i}\left( \frac{w_{i}}{c_{i}};\tilde{\theta}\right)  \\
&=&1-\left( 1-\left( 1-\frac{w_{i}}{c_{i}}\right) ^{\beta _{i}}\right)  \\
&=&\left( 1-\frac{w_{i}}{c_{i}}\right) ^{\beta _{i}}
\end{eqnarray*}%
We then deduce that 
\[
\beta _{i}=\frac{\log \delta _{i}}{\log \left( 1-\frac{w_{i}}{c_{i}}\right) }
\]%
The marginal cdf are%
\[
F\left( x;\tilde{\theta}\right) =1-\left( 1-x\right) ^{\frac{\log \delta _{i}%
}{\log \left( 1-\frac{w_{i}}{c_{i}}\right) }}
\]%
The inverse marginal cdf is%
\[
F^{-1}\left( u_{i};\tilde{\theta}\right) =1-\left( 1-u_{i}\right) ^{\frac{%
\log \left( 1-\frac{w_{i}}{c_{i}}\right) }{\log \delta _{i}}}
\]%
Taking derivatives, we have%
\begin{eqnarray*}
\frac{\partial F^{-1}\left( u_{i};\tilde{\theta}\right) }{\partial c_{i}}
&=&-\left( 1-u_{i}\right) ^{\frac{\log \left( 1-\frac{w_{i}}{c_{i}}\right) }{%
\log \delta _{i}}}\frac{w_{i}\log \left( 1-u_{i}\right) }{c_{i}^{2}\log
\delta _{i}}\left( 1-\frac{w_{i}}{c_{i}}\right) ^{-1} \\
\frac{\partial F^{-1}\left( u_{i};\tilde{\theta}\right) }{\partial c_{j}}
&=&0
\end{eqnarray*}%
Now for the whole loss%
\begin{eqnarray*}
&&\frac{\partial }{\partial c_{i}}E\left[ \sum_{k=1}^{n}\min \left(
c_{k}F_{k}^{-1}\left( u_{k};c_{k}\right) ,w_{k}\right) +\sum_{k=1}^{n}\left(
1+\gamma \right) \left( c_{k}F_{k}^{-1}\left( u_{k};c_{k}\right)
-w_{k}\right) u_{k}\left( x\right) \right]  \\
&&E\left[ \sum_{k=1}^{n}\frac{\partial }{\partial c_{i}}\left\{ \min \left(
c_{k}F_{k}^{-1}\left( u_{k};c_{k}\right) ,w_{k}\right) \right\}
+\sum_{k=1}^{n}\left( 1+\gamma \right) \frac{\partial }{\partial c_{i}}%
\left\{ \left( c_{k}F_{k}^{-1}\left( u_{k};c_{k}\right) -w_{k}\right)
u_{k}\left( x\right) \right\} \right]  \\
&&E\left[ \sum_{k=1}^{n}\frac{\partial }{\partial c_{i}}\left\{ \min \left(
c_{k}F_{k}^{-1}\left( u_{k};c_{k}\right) ,w_{k}\right) \right\}
+\sum_{k=1}^{n}\left( 1+\gamma \right) \left[ \frac{\partial }{\partial c_{i}%
}\left\{ \left( c_{k}F_{k}^{-1}\left( u_{k};c_{k}\right) -w_{k}\right)
\right\} u_{k}\left( x\right) +\left( c_{k}F_{k}^{-1}\left(
u_{k};c_{k}\right) -w_{k}\right) \frac{\partial }{\partial c_{i}}\left\{
u_{k}\left( x\right) \right\} \right] \right]  \\
&&E\left[ \sum_{k=1}^{n}\frac{\partial \left[ c_{k}F_{k}^{-1}\left(
u_{k};c_{k}\right) \right] }{\partial c_{i}}1\left\{ c_{k}F_{k}^{-1}\left(
u_{k};c_{k}\right) <w_{k}\right\} +\left( 1+\gamma \right) \sum_{k=1}^{n}%
\frac{\partial \left[ c_{k}F_{k}^{-1}\left( u_{k};c_{k}\right) \right] }{%
\partial c_{i}}u_{k}\left( x\right) \right]  \\
&&E\left[ \sum_{k=1}^{n}\left\{ 1\left\{ c_{k}x_{k}<w_{k}\right\} +\left(
1+\gamma \right) u_{k}\left( x\right) \right\} \left( c_{k}\frac{\partial
F_{k}^{-1}\left( u_{k};c_{k}\right) }{\partial c_{i}}+F_{k}^{-1}\left(
u_{k};c_{k}\right) \frac{\partial c_{k}}{\partial c_{i}}\right) \right]  \\
&&E\left[ \sum_{k=1}^{n}\left\{ 1\left\{ c_{k}x_{k}<w_{k}\right\} +\left(
1+\gamma \right) u_{k}\left( x\right) \right\} \left( c_{k}\frac{\partial
F_{k}^{-1}\left( u_{k};c_{k}\right) }{\partial c_{k}}+x_{k}\right) \right] 
\end{eqnarray*}%
\begin{eqnarray*}
\frac{\partial }{\partial c_{i}}\left\{ \left( c_{k}F_{k}^{-1}\left(
u_{k};c_{k}\right) -w_{k}\right) \right\}  &=&\frac{\partial }{\partial c_{i}%
}c_{k}F_{k}^{-1}\left( u_{k};c_{k}\right)  \\
&=&c_{k}\frac{\partial F_{k}^{-1}\left( u_{k};c_{k}\right) }{\partial c_{i}}%
+x_{k}\frac{\partial c_{k}}{\partial c_{i}} \\
&=&%
\begin{array}{ccc}
-\left( 1-u_{i}\right) ^{\frac{\log \left( 1-\frac{w_{i}}{c_{i}}\right) }{%
\log \delta _{i}}}\frac{w_{i}\log \left( 1-u_{i}\right) }{c_{i}\log \delta
_{i}}\left( 1-\frac{w_{i}}{c_{i}}\right) ^{-1}+F_{i}^{-1}\left(
u_{i};c_{i}\right)  & , & i=k \\ 
0 & , & \text{otherwise}%
\end{array}
\\
\frac{\partial }{\partial c_{i}}\left\{ \min \left( c_{k}F_{k}^{-1}\left(
u_{k};c_{k}\right) ,w_{k}\right) \right\}  &=&%
\begin{array}{ccc}
F_{k}^{-1}\left( u_{k};c_{k}\right) +c_{k}\frac{\partial }{\partial c_{i}}%
F_{k}^{-1}\left( u_{k};c_{k}\right)  & , & c_{k}F_{k}^{-1}\left(
u_{k};c_{k}\right) <w_{k} \\ 
0 & , & \text{otherwise}%
\end{array}
\\
&=&%
\begin{array}{ccc}
x_{k}+c_{k}\frac{\partial }{\partial c_{i}}F_{k}^{-1}\left(
u_{k};c_{k}\right)  & , & c_{k}x_{k}<w_{k} \\ 
0 & , & \text{otherwise}%
\end{array}
\\
\frac{\partial u_{k}\left( c\circ x\right) }{\partial c_{j}} &=&0
\end{eqnarray*}%
\begin{eqnarray*}
\frac{\partial \left[ \left( c_{k}x_{k}-w_{k}\right) u_{k}\left( c\circ
x\right) \right] }{\partial c_{j}} &=&u_{k}\left( c\circ x\right) \frac{%
\partial \left( c_{k}x_{k}-w_{k}\right) }{\partial c_{j}}+\left(
c_{k}x_{k}-w_{k}\right) \frac{\partial u_{k}\left( c\circ x\right) }{%
\partial c_{j}} \\
\frac{\partial \left( c_{k}x_{k}-w_{k}\right) }{\partial c_{j}} &=&x_{k}%
\frac{\partial c_{k}}{\partial c_{j}}=\left\{ 
\begin{array}{ccc}
x_{k} & , & j=k \\ 
0 & , & otherwise%
\end{array}%
\right.  \\
\frac{\partial u_{k}\left( c\circ x\right) }{\partial c_{j}} &=&0
\end{eqnarray*}%
\begin{eqnarray*}
\left( \nabla _{c}L\right) _{i} &=&\left\{ x_{i}+c_{i}\frac{\partial
F_{i}^{-1}\left( u_{i};c_{i}\right) }{\partial c_{i}}\right\} 1\left\{
c_{i}x_{i}<w_{i}\right\} +\left\{ x_{i}+c_{i}\frac{\partial F_{k}^{-1}\left(
u_{i};c_{i}\right) }{\partial c_{i}}\right\}  \\
&=&\left( 1\left\{ c_{i}x_{i}<w_{i}\right\} +1\right) \left\{ x_{i}+c_{i}%
\frac{\partial F_{k}^{-1}\left( u_{i};c_{i}\right) }{\partial c_{i}}\right\} 
\end{eqnarray*}%
\begin{eqnarray*}
\frac{\partial \left[ e_{k}\left( I_{D}-\left( 1+\gamma \right) A_{D}\right)
^{-1}1_{D}\right] }{\partial A_{D}} &=&\left( 1+\gamma \right) \left(
I_{D}-\left( 1+\gamma \right) A_{D}\right) ^{-T}e_{k}^{T}1_{D}^{T}\left(
I_{D}-\left( 1+\gamma \right) A_{D}\right) ^{-T} \\
&=&\left( 1+\gamma \right) \left[ e_{k}\left( I_{D}-\left( 1+\gamma \right)
A_{D}\right) ^{-1}\right] ^{T}\left[ \left( I_{D}-\left( 1+\gamma \right)
A_{D}\right) ^{-1}1_{D}\right] ^{T} \\
&=&\left( 1+\gamma \right) \left( I_{D}-\left( 1+\gamma \right) A_{D}\right)
^{-T}e_{k}^{T}u_{D}^{T}
\end{eqnarray*}

\section{Old}

We would like to compute the gradient of expected losses%
\[
\nabla _{A}\left\{ E\left[ \sum_{k=1}^{n}\min \left( x_{k},w_{k}\right)
+\sum_{k=1}^{n}\left( x_{k}-w_{k}\right) u_{k}\left( x\right) \right]
\right\} 
\]%
where $\nabla _{A}$ means the gradient with respect to the entries of the
matrix $A$. Since $x_{k}$ and $w_{k}$ do not depend on $A$, and expectations
are linear,%
\[
\nabla _{A}\left\{ E\left[ \sum_{k=1}^{n}\min \left( x_{k},w_{k}\right)
+\sum_{k=1}^{n}\left( x_{k}-w_{k}\right) u_{k}\left( x\right) \right]
\right\} =E\sum_{k=1}^{n}\left( x_{k}-w_{k}\right) \nabla _{A}u_{k}\left(
x\right) 
\]%
To compute $\nabla _{A}u_{k}\left( x\right) $, we use that%
\[
u_{k}\left( x\right) =\left\{ 
\begin{array}{ccc}
0 & , & \text{if }k\notin D\left( A\right)  \\ 
e_{k}\left( I_{D}-\left( 1+\gamma \right) A_{D}\right) ^{-1}1_{D} & , & 
\text{if }k\in D\left( A\right) 
\end{array}%
\right. 
\]%
where $D$ is the set of defaulting nodes, $I_{D}$ is the $\left\vert
D\right\vert \times \left\vert D\right\vert $ identity matrix, $e_{k}$ is a
row vector with a $1$ in column $k$ and zeros otherwise, and $A_{D}$ is the
matrix $A$ restricted to $D$. We can write%
\[
A_{D}=S^{\prime }AS
\]%
for a matrix $S$ with dimension $n\times \left\vert D\right\vert $ 
\[
S_{ij}=\left\{ 
\begin{array}{ccc}
1 & , & \text{if }j\in D\text{ and }i\leq j \\ 
0 & , & \text{otherwise}%
\end{array}%
\right. 
\]%
so that $A_{D}$ has dimension $\left\vert D\right\vert \times \left\vert
D\right\vert $. For example, if $n=3$ and $D=\left\{ 1,3\right\} $ 
\[
S=\left[ 
\begin{array}{cc}
1 & 0 \\ 
0 & 0 \\ 
0 & 1%
\end{array}%
\right] 
\]%
and%
\begin{eqnarray*}
A_{D} &=&\left[ 
\begin{array}{ccc}
1 & 0 & 0 \\ 
0 & 0 & 1%
\end{array}%
\right] \left[ 
\begin{array}{ccc}
a_{11} & a_{12} & a_{13} \\ 
a_{21} & a_{22} & a_{23} \\ 
a_{31} & a_{32} & a_{33}%
\end{array}%
\right] \left[ 
\begin{array}{cc}
1 & 0 \\ 
0 & 0 \\ 
0 & 1%
\end{array}%
\right]  \\
&=&\left[ 
\begin{array}{cc}
a_{11} & a_{13} \\ 
a_{31} & a_{33}%
\end{array}%
\right] 
\end{eqnarray*}%
We also note that $D$ depends on $A$ 
\begin{eqnarray*}
D\left( A\right)  &=&\left\{ i:p_{i}\left( A\right) <\bar{p}_{i}\right\}  \\
&=&\left\{ i:\min \left\{ \bar{p}_{i},\max \left\{ \sum_{j}p_{j}\left(
A\right) a_{ji}+c_{i}-x_{i},0\right\} \right\} <\bar{p}_{i}\right\} 
\end{eqnarray*}%
However, $p_{i}\left( A\right) $ is continuous in $A$, so small changes in $A
$ can only affect $D$ if%
\[
p_{i}\left( A\right) =\bar{p}_{i}
\]%
We now compute%
\[
\nabla _{A}u_{k}\left( x\right) =\left\{ 
\begin{array}{ccc}
0 & , & \text{if }k\notin D\left( A\right)  \\ 
\nabla _{A}e_{k}\left( I_{D}-S^{T}AS\right) ^{-1}1_{D} & , & \text{if }k\in
D\left( A\right) 
\end{array}%
\right. 
\]%
Use the formulas%
\begin{eqnarray*}
\frac{\partial a^{T}X^{-1}b}{\partial X} &=&-X^{-T}ab^{T}X^{-T} \\
\partial \left( XY\right)  &=&\left( \partial X\right) Y+X\partial Y \\
\partial \left( S^{T}AS\right)  &=&\partial \left( S^{T}A\right) S+\left(
S^{T}A\right) \partial S \\
&=&\left[ \partial \left( S^{T}\right) A+S^{T}\partial A\right] S+\left(
S^{T}A\right) \partial S \\
&=&S^{T}\left( \partial A\right) S
\end{eqnarray*}%
to get%
\begin{eqnarray*}
\frac{\partial \left[ e_{k}\left( I_{D}-\left( 1+\gamma \right) A_{D}\right)
^{-1}1_{D}\right] }{\partial A_{D}} &=&\left( 1+\gamma \right) \left(
I_{D}-\left( 1+\gamma \right) A_{D}\right) ^{-T}e_{k}^{T}1_{D}^{T}\left(
I_{D}-\left( 1+\gamma \right) A_{D}\right) ^{-T} \\
&=&\left( 1+\gamma \right) \left[ e_{k}\left( I_{D}-\left( 1+\gamma \right)
A_{D}\right) ^{-1}\right] ^{T}\left[ \left( I_{D}-\left( 1+\gamma \right)
A_{D}\right) ^{-1}1_{D}\right] ^{T} \\
&=&\left( 1+\gamma \right) \left( I_{D}-\left( 1+\gamma \right) A_{D}\right)
^{-T}e_{k}^{T}u_{D}^{T}
\end{eqnarray*}%
and 
\[
\frac{\partial \left[ e_{k}\left( I_{D}-A_{D}\right) ^{-1}1_{D}\right] }{%
\partial A_{D^{c}}}=0
\]%
where $D^{c}$ is the complement of $D$.%
\[
\nabla _{A}u_{k}\left( x\right) =S\left[ \nabla _{A_{D}}u_{k}\left( x\right) %
\right] S^{T}
\]%
Finally, 
\[
E\sum_{k=1}^{n}\left( x_{k}-w_{k}\right) S\left[ \nabla _{A_{D}}u_{k}\left(
x\right) \right] S^{T}
\]%
Now we compute the gradient with respect to $c$, assuming that $x_{k}$ is a
random variable between $0$ and $1$ and thus total losses are%
\[
E\left[ \sum_{k=1}^{n}\min \left( c_{k}x_{k},w_{k}\right)
+\sum_{k=1}^{n}\left( c_{k}x_{k}-w_{k}\right) u_{k}\left( c\circ x\right) %
\right] 
\]%
where $c\circ x$ denotes element-wise multiplication.

We compute%
\[
\nabla _{c}\left\{ E\left[ \sum_{k=1}^{n}\min \left( c_{k}x_{k},w_{k}\right)
+\sum_{k=1}^{n}\left( c_{k}x_{k}-w_{k}\right) u_{k}\left( c\circ x\right) %
\right] \right\} 
\]%
Using%
\begin{eqnarray*}
\frac{\partial \min \left( c_{k}x_{k},w_{k}\right) }{\partial c_{j}}
&=&\left\{ 
\begin{array}{ccc}
x_{k} & , & c_{k}<\frac{w_{k}}{x_{k}}\text{ and }j=k \\ 
0 & , & otherwise%
\end{array}%
\right.  \\
\nabla _{c}\left\{ \min \left( c_{k}x_{k},w_{k}\right) \right\}  &=&\left( 
\begin{array}{ccccc}
0 & \cdots  & 1_{\left\{ c_{k}<\frac{w_{k}}{x_{k}}\right\} }x_{k} & \cdots 
& 0%
\end{array}%
\right)  \\
\sum_{k=1}^{n}\nabla _{c}\left\{ \min \left( c_{k}x_{k},w_{k}\right)
\right\}  &=&\left( 
\begin{array}{ccccc}
1_{\left\{ c_{1}<\frac{w_{1}}{x_{1}}\right\} }x_{1} & \cdots  & 1_{\left\{
c_{k}<\frac{w_{k}}{x_{k}}\right\} }x_{k} & \cdots  & 1_{\left\{ c_{n}<\frac{%
w_{n}}{x_{n}}\right\} }x_{n}%
\end{array}%
\right) 
\end{eqnarray*}%
and%
\begin{eqnarray*}
\frac{\partial \left[ \left( c_{k}x_{k}-w_{k}\right) u_{k}\left( c\circ
x\right) \right] }{\partial c_{j}} &=&u_{k}\left( c\circ x\right) \frac{%
\partial \left( c_{k}x_{k}-w_{k}\right) }{\partial c_{j}}+\left(
c_{k}x_{k}-w_{k}\right) \frac{\partial u_{k}\left( c\circ x\right) }{%
\partial c_{j}} \\
\frac{\partial \left( c_{k}x_{k}-w_{k}\right) }{\partial c_{j}} &=&x_{k}%
\frac{\partial c_{k}}{\partial c_{j}}=\left\{ 
\begin{array}{ccc}
x_{k} & , & j=k \\ 
0 & , & otherwise%
\end{array}%
\right.  \\
\frac{\partial u_{k}\left( c\circ x\right) }{\partial c_{j}} &=&0
\end{eqnarray*}%
we get%
\begin{eqnarray*}
&&\nabla _{c}\left\{ E\left[ \sum_{k=1}^{n}\min \left(
c_{k}x_{k},w_{k}\right) +\sum_{k=1}^{n}\left( c_{k}x_{k}-w_{k}\right)
u_{k}\left( c\circ x\right) \right] \right\}  \\
&=&E\left[ \sum_{k=1}^{n}\nabla _{c}\min \left( c_{k}x_{k},w_{k}\right)
+\sum_{k=1}^{n}\nabla _{c}\left[ \left( c_{k}x_{k}-w_{k}\right) u_{k}\left(
c\circ x\right) \right] \right]  \\
&=&E\left[ 
\begin{array}{c}
\left( 
\begin{array}{ccccc}
1_{\left\{ c_{1}<\frac{w_{1}}{x_{1}}\right\} }x_{1} & \cdots  & 1_{\left\{
c_{k}<\frac{w_{k}}{x_{k}}\right\} }x_{k} & \cdots  & 1_{\left\{ c_{n}<\frac{%
w_{n}}{x_{n}}\right\} }x_{n}%
\end{array}%
\right)  \\ 
+\left( 
\begin{array}{ccccc}
x_{1}u_{1}\left( c\circ x\right)  & \cdots  & x_{k}u_{k}\left( c\circ
x\right)  & \cdots  & x_{n}u_{n}\left( c\circ x\right) 
\end{array}%
\right) 
\end{array}%
\right]  \\
&=&E\left[ \left( 
\begin{array}{ccccc}
\left[ 1_{\left\{ c_{1}<\frac{w_{1}}{x_{1}}\right\} }+u_{1}\left( c\circ
x\right) \right] x_{1} & \cdots  & \left[ 1_{\left\{ c_{k}<\frac{w_{k}}{x_{k}%
}\right\} }+u_{k}\left( c\circ x\right) \right] x_{k} & \cdots  & \left[
1_{\left\{ c_{n}<\frac{w_{n}}{x_{n}}\right\} }+u_{n}\left( c\circ x\right) %
\right] x_{n}%
\end{array}%
\right) \right] 
\end{eqnarray*}%
We pick the parameter $\beta $ of the beta distribution with pdf%
\[
f\left( x\right) =\frac{1}{B\left( \alpha ,\beta \right) }x^{\alpha
-1}x^{\beta -1}
\]%
and cdf%
\[
F\left( x\right) =\frac{1}{B\left( \alpha ,\beta \right) }%
\int_{0}^{x}x^{\alpha -1}x^{\beta -1}dx
\]%
so that

\begin{eqnarray*}
\delta  &=&P\left( cx>w\right)  \\
&=&1-F\left( \frac{w}{c},\alpha ,\beta \right)  \\
&=&1-\frac{1}{B\left( \alpha ,\beta \right) }\int_{0}^{\frac{w}{c}}x^{\alpha
-1}x^{\beta -1}dx
\end{eqnarray*}%
Solving for $\beta $ gives a solution%
\[
\beta =\beta \left( c,w,\delta ,\alpha \right) 
\]%
The CDF and PDF\ are then%
\begin{eqnarray*}
F\left( x;c\right)  &=&\frac{1}{B\left( \alpha ,\beta \left( c,w,\delta
,\alpha \right) \right) }\int_{0}^{x}x^{\alpha -1}x^{\beta \left( c,w,\delta
,\alpha \right) -1}dx \\
f\left( x;c\right)  &=&\frac{1}{B\left( \alpha ,\beta \left( c,w,\delta
,\alpha \right) \right) }x^{\alpha -1}x^{\beta \left( c,w,\delta ,\alpha
\right) -1}
\end{eqnarray*}%
To compute 
\[
\nabla _{c}\left\{ E\left[ \sum_{k=1}^{n}\min \left( c_{k}x_{k},w_{k}\right)
+\sum_{k=1}^{n}\left( c_{k}x_{k}-w_{k}\right) u_{k}\left( c\circ x\right) %
\right] \right\} 
\]%
we then note that%
\begin{eqnarray*}
&&\nabla _{c}\left\{ E\left[ \sum_{k=1}^{n}\min \left(
c_{k}x_{k},w_{k}\right) +\sum_{k=1}^{n}\left( c_{k}x_{k}-w_{k}\right)
u_{k}\left( c\circ x\right) \right] \right\}  \\
&=&\nabla _{c}\int_{0}^{1}...\int_{0}^{1}\left[ \sum_{k=1}^{n}\min \left(
c_{k}x_{k},w_{k}\right) +\sum_{k=1}^{n}\left( c_{k}x_{k}-w_{k}\right)
u_{k}\left( c\circ x\right) \right] f\left( x;c\right) dx_{1}...dx_{n} \\
&=&\int_{0}^{1}...\int_{0}^{1}\nabla _{c}\left\{ \left[ \sum_{k=1}^{n}\min
\left( c_{k}x_{k},w_{k}\right) +\sum_{k=1}^{n}\left( c_{k}x_{k}-w_{k}\right)
u_{k}\left( c\circ x\right) \right] f\left( x;c\right) \right\}
dx_{1}...dx_{n} \\
&=&\int_{0}^{1}...\int_{0}^{1}\nabla _{c}\left[ \sum_{k=1}^{n}\min \left(
c_{k}x_{k},w_{k}\right) +\sum_{k=1}^{n}\left( c_{k}x_{k}-w_{k}\right)
u_{k}\left( c\circ x\right) \right] f\left( x;c\right) dx_{1}...dx_{n} \\
&&+\int_{0}^{1}...\int_{0}^{1}\left[ \sum_{k=1}^{n}\min \left(
c_{k}x_{k},w_{k}\right) +\sum_{k=1}^{n}\left( c_{k}x_{k}-w_{k}\right)
u_{k}\left( c\circ x\right) \right] \nabla _{c}f\left( x;c\right)
dx_{1}...dx_{n} \\
&=&\int_{0}^{1}...\int_{0}^{1}\nabla _{c}\left[ \sum_{k=1}^{n}\min \left(
c_{k}x_{k},w_{k}\right) +\sum_{k=1}^{n}\left( c_{k}x_{k}-w_{k}\right)
u_{k}\left( c\circ x\right) \right] f\left( x;c\right) dx_{1}...dx_{n} \\
&&+\int_{0}^{1}...\int_{0}^{1}L\left( x\right) \frac{\nabla _{c}f\left(
x;c\right) }{f\left( x;c\right) }f\left( x;c\right) dx_{1}...dx_{n} \\
&=&E\left[ \nabla _{c}\left[ \sum_{k=1}^{n}\min \left(
c_{k}x_{k},w_{k}\right) +\sum_{k=1}^{n}\left( c_{k}x_{k}-w_{k}\right)
u_{k}\left( c\circ x\right) \right] \right] +E\left[ L\left( x\right) \frac{%
\nabla _{c}f\left( x;c\right) }{f\left( x;c\right) }\right] 
\end{eqnarray*}%
For independent random variables $x$%
\[
f\left( x;c\right) =\dprod\limits_{k=1}^{n}f\left( x_{k};c_{k}\right) 
\]%
so that%
\[
\frac{\nabla _{c}f\left( x;c\right) }{f\left( x;c\right) }=\left( \frac{1}{%
f\left( x_{1};c_{1}\right) }\frac{df\left( x_{1};c_{1}\right) }{dc_{1}},..,%
\frac{1}{f\left( x_{k};c_{k}\right) }\frac{df\left( x_{k};c_{k}\right) }{%
dc_{k}},..,\frac{1}{f\left( x_{n};c_{n}\right) }\frac{df\left(
x_{n};c_{n}\right) }{dc_{n}}\right) 
\]%
Assume $\alpha =1$. Then%
\begin{eqnarray*}
\delta  &=&1-\frac{1}{B\left( 1,\beta \right) }\int_{0}^{\frac{w}{c}%
}x^{\beta -1}dx \\
\delta  &=&\left( 1-\frac{w}{c}\right) ^{\beta } \\
\beta  &=&\frac{\log \delta }{\log \left( 1-\frac{w}{c}\right) }
\end{eqnarray*}%
and%
\begin{eqnarray*}
f_{k}\left( x;1,\beta \right)  &=&\beta \left( 1-x\right) ^{\beta -1} \\
f_{k}\left( x;1,\frac{\log \delta }{\log \left( 1-\frac{w}{c}\right) }%
\right)  &=&\frac{\log \delta }{\log \left( 1-\frac{w}{c}\right) }\left(
1-x\right) ^{\frac{\log \delta }{\log \left( 1-\frac{w}{c}\right) }-1} \\
\frac{1}{f\left( x_{k};c_{k}\right) }\frac{df\left( x_{k};c_{k}\right) }{%
dc_{k}} &=&\frac{1}{\frac{\log \delta }{\log \left( 1-\frac{w}{c}\right) }%
\left( 1-x\right) ^{\frac{\log \delta }{\log \left( 1-\frac{w}{c}\right) }-1}%
}\left( \left\{ \frac{\log \delta }{\log \left( 1-\frac{w}{c}\right) }%
\right\} \frac{d}{dc}\left\{ \left( 1-x\right) ^{\frac{\log \delta }{\log
\left( 1-\frac{w}{c}\right) }-1}\right\} +\left\{ \left( 1-x\right) ^{\frac{%
\log \delta }{\log \left( 1-\frac{w}{c}\right) }-1}\right\} \frac{d}{dc}%
\left\{ \frac{\log \delta }{\log \left( 1-\frac{w}{c}\right) }\right\}
\right)  \\
&=&\frac{1}{\left( 1-x\right) ^{\frac{\log \delta }{\log \left( 1-\frac{w}{c}%
\right) }}}\frac{d}{dc}\left\{ \left( 1-x\right) ^{\frac{\log \delta }{\log
\left( 1-\frac{w}{c}\right) }}\right\} -\frac{1}{\log \left( 1-\frac{w}{c}%
\right) }\frac{1}{\left( c-w\right) }\frac{w}{c} \\
&=&-\frac{w\left( \log \left( 1-\frac{w}{c}\right) +\log \left( 1-x\right)
\log \left( \delta \right) \right) }{c\left( c-w\right) \log \left( 1-\frac{w%
}{c}\right) ^{2}}
\end{eqnarray*}

If we use a gaussian copula with correlation matrix $R$ and marginals 
\[
f\left( x_{k};c_{k}\right) =\frac{1}{B\left( \alpha _{k},\beta \left(
c_{k},w_{k},\delta _{k},\alpha _{k}\right) \right) }x^{\alpha
_{k}-1}x^{\beta \left( c_{k},w_{k},\delta _{k},\alpha _{k}\right) -1}
\]%
the density of the copula is%
\[
c_{R}^{Gauss}\left( u\right) =\frac{1}{\sqrt{\det R}}\exp \left( -\frac{1}{2}%
\left( 
\begin{array}{c}
\Phi ^{-1}\left( u_{1}\right)  \\ 
... \\ 
\Phi ^{-1}\left( u_{n}\right) 
\end{array}%
\right) ^{T}\left( R^{-1}-I\right) \left( 
\begin{array}{c}
\Phi ^{-1}\left( u_{1}\right)  \\ 
... \\ 
\Phi ^{-1}\left( u_{n}\right) 
\end{array}%
\right) \right) 
\]%
Thus,%
\[
f\left( x;c\right) =c_{R}^{Gauss}\left( F\left( x_{1};c_{1}\right)
,...,F\left( x_{n};c_{n}\right) \right) f\left( x_{1};c_{1}\right)
...f\left( x_{n};c_{n}\right) 
\]%
We can now compute the $i^{th}$ component of $\frac{\nabla _{c}f\left(
x;c\right) }{f\left( x;c\right) }:$ 
\[
c_{R,k}^{Gauss}\left( F\left( x_{1};c_{1}\right) ,...,F\left(
x_{n};c_{n}\right) \right) f\left( x_{k};c_{k}\right) \frac{df\left(
x_{k};c_{k}\right) }{dc_{k}}+c_{R}^{Gauss}\left( F\left( x_{1};c_{1}\right)
,...,F\left( x_{n};c_{n}\right) \right) \frac{1}{f\left( x_{k};c_{k}\right) }%
\frac{df\left( x_{k};c_{k}\right) }{dc_{k}}
\]

\bigskip 

\bigskip 

Try%
\begin{eqnarray*}
&&\nabla _{c}\left\{ E\left[ \sum_{k=1}^{n}\min \left( c_{k}F^{-1}\left(
u_{k};c_{k}\right) ,w_{k}\right) +\sum_{k=1}^{n}\left( c_{k}F^{-1}\left(
u_{k};c_{k}\right) -w_{k}\right) u_{k}\left( c\circ F^{-1}\left( x\right)
\right) \right] \right\}  \\
&=&\nabla _{c}\left\{ E\left[ \sum_{k=1}^{n}\min \left(
c_{k}x_{k},w_{k}\right) +\sum_{k=1}^{n}\left( c_{k}x_{k}-w_{k}\right)
u_{k}\left( c\circ x\right) \right] \right\}  \\
&&+\left\{ E\left[ \sum_{k=1}^{n}\min \left( c_{k}\nabla _{c}F^{-1}\left(
u_{k};c_{k}\right) ,w_{k}\right) +\sum_{k=1}^{n}\left( c_{k}\nabla
_{c}F^{-1}\left( u_{k};c_{k}\right) -w_{k}\right) u_{k}\left( c\circ
F^{-1}\left( x\right) \right) \right] \right\} 
\end{eqnarray*}

\end{document}
