By using detailed data on balance sheet exposures for US financial firms, we have constructed a measure of network spillovers that arise through default cascades in the period 2002-2016. We find that default spillovers, on their own, can amplify expected losses by up to $25\%$ during the financial crisis, but are close to zero before 2008 and after 2012. Default spillovers can be large when nodes inside the network are more exposed to losses outside the network or when the topology of the network implies a higher degree of connectivity among nodes. We find that both elements are important contributors to the time-series dynamics of spillovers and that they can move together or in opposite directions depending the time period examined. In contrast to some narratives of the crisis, we find that neither the exposure to the outside sector nor the connectivity of the financial network increased before the financial crisis. Instead, we find that the events \textit{during} the crisis made the network fragile. After the crisis, our measure of spillovers returned to its low pre-crisis levels, although in the last two years of our sample (2015 and 2016) it has shown a slight increase that may provide a useful signal for policymakers. Considering further amplification mechanisms, such as bankruptcy costs, exacerbates the magnitude of default spillover losses but does not change the conclusion that spillovers were important in 2008-2012 and negligible in the rest of our sample. 