%% LyX 2.2.2 created this file.  For more info, see http://www.lyx.org/.
%% Do not edit unless you really know what you are doing.
\documentclass[english]{article}
\usepackage[T1]{fontenc}
\usepackage[latin9]{inputenc}
\usepackage{babel}
\begin{document}

\title{Network Contagion To-Dos and Comments}

\author{Collin Jones}
\maketitle

\section*{To-Dos}
\begin{itemize}
\item Read through paper and update anything deprecated (hasn't been done
in a little while)
\item Use new FOCUS reports data to refine our dealer nodes (at the very
least, get specific $\beta$ values for Goldman Sachs and Morgan Stanley,
to eliminate the need for the aggregate dealer nodes).
\item Improve the catch-all 'Other' category, or remove it entirely. We're
convinced that the firms being selected as representative of that
sector's default probability are not great representations.This will
change the final NVI values quite a bit, as that category has the
highest default probabilities now. Could start at flow of funds documentation
and see where they get their data for some of the sectors we lump
into 'Other'. 
\begin{itemize}
\item One idea that we discussed is to only assign the 'Other' category
a magnitude of assets equaling the total assets of firms selected
to fit that category from the KMV dataset. This will prevent us from
assigning huge numbers of assets to a relatively small sample of firms
(which is what we currently do). 
\end{itemize}
\item Address Joao's comments:
\begin{itemize}
\item Do worst case scenario over whole sample (e.g., for each bank, pick
worst prob of default over sample and assign that prob for the whole
sample) 
\item Do not speculate about mortgages becoming riskier as explanation 
\item Must explain why a simple weighted average of prob of default does
not tell the whole story \textendash{} otherwise, why do we need the
whole NVI and all the work? Maybe not even show that graph in main
text since it invites these questions 
\item Just like we plot a case with constant delta, can we plot a case where
the effect of delta moving but nothing else moving is shown? 
\item The benchmark should be that non-insured deposits are outside the
network, then do robustness (since most are corporate deposits) 
\item The small network example is good, keep using those numbers to illustrate
throughout 
\item When we define $\delta_{i}$, it is confusing (need to make clearer
it is just due to shocks) 
\item When we introduce contagion index, must explain that it is equal to
dollar amount of intranetwork liabilities; then explain why it is
increasing in each of the three components (for example, why is the
contagion index increasing in net worth? That doesn\textquoteright t
seem to make sense unless you explain high net worth reduces contagion
through lambda, but given lambda, it increases it due to size, but
it cancels anyway) 
\item When we say \textquotedblleft We define a frim\textquoteright s quarterly
EDF measure to be the average of its daily measures over a given quarter.\textquotedblright ,
it is confusing since it is not clear if it is quarterly prob of default
or annual prob of default at quarterly frequency (and why pick annual
EDF?) 
\item Maybe acknowledge that NVI does not predict rising vulnerability pre-crisis,
or compare to how it does with other measures (like CoVaR, srisk)
{[}{[}Note: When I ran some cross-sectional predictive regressions
for my fire-sale project and included the contagion index, it did
really well, we should go back and look at those results{]}{]} 
\item When we show bank-specific plots, mention that acquiring a financial
firm does increase outside assets if the acquired firm has outside
assets 
\item How much does size matter? Measure seems to be all ratios (beta, NVI,
etc are ratios). Most systemic risk measures boil down to size, but
ours does not seem to do so, which is nice if we can document it 
\item Default is very binary: you either default and haircut liabilities,
or you don\textquoteright t. In the real world, as prob of default
increases, there are consequences/changes. Risk goes up. Value of
liabilities goes down, even without default. Counterparties do not
roll over debt. By the time you default, a lot of the action may have
already happened, and the measure also misses all of it. Can we somehow
incorporate that? Or think about that? {[}{[}Note: can use the glasserman
\textquotedblleft loss of confidence\textquotedblright{} case?{]}{]} 
\item What defines who\textquoteright s inside and outside the financial
network in the model? How much does it matter both theoretically and
empirically? 
\item Why do you break down Top 10 and Top 25 broker-dealers in the contribution
by sector plot (or in the weighted avg of default)? They are all broker
dealers after all
\end{itemize}
\item Simulations
\begin{itemize}
\item Decide where this all fits in the paper.
\item Run and interpret simulation results with different bankruptcy costs.
\item Run simulations that minimize the different between the NVI and the
actual ratio of expected losses (as tight as we can get)
\item Re-derive the gradient for the beta distribution where $\alpha=2$.
I was having some trouble making this work in closed-form.
\end{itemize}
\end{itemize}

\section*{Comments}

What follows are comments and explanations of several of the more-nuanced
data decisions made in the process of writing the paper, which may
be difficult to glean solely from looking at the code. I would recommend
reviewing this section before undertaking any major overhauls of the
code, to at least see why things are set up their current way.
\begin{itemize}
\item The $\beta^{+}$for the benchmark NVI is selected from a sample of
22 major system nodes - 20 belong to major BHCs in the FR-Y9C sample,
and 2 are broker-dealer aggregate nodes compile from SIFMA aggregate
FOCUS report data (Model\_series\_processing.do).
\item We currently use KMV versions 9 default probability data. Using version
8 instead causes some substantial differences (Update\_Data.do). 
\item Matching between the FR-Y9C to the KMV dataset has a number of steps.
RSSID numbers for BHCs are taken from the FR-Y9C data from FI. An
RSSID-Permco matching is taken from data posted online by FI. A Permco-CUSIP
match is pulled from CRSP, which we access through WRDS. The CUSIP
is provided in the KMV dataset, so this is merged to find the MKMVID
(Match\_RSSID\_MKMVID.do).
\item Early in the project, we had a problem where the quantity of uninsured
deposits estimated in a BHC's commercial bank Call Report filings
exceeded the value of domestic deposits in the parent BHC's FR-Y9C
filings. This led to strange negative values of insured Y9C deposits.
To fix this issue, we now take the estimate of insured deposits from
the Call Report, then subtract that from the FR-Y9C domestic deposits
numbers to obtain an estimate for uninsured deposits (CallDeposits.do
and Analysis\_Y9C.do).
\item {[}Call Report matching{]}
\item Analysis\_Y9C.do
\begin{itemize}
\item The KMV default probability data comes at a daily frequency, which
must be aggregated to the quarterly level to use with our balance
sheet data. We currently use the end-of-quarter value, but have used
the mean in past iterations. (Analysis\_Y9C.do)
\item We remove KMV default probability data of firms that have filed for
bankruptcy, according to Moody's drd database. If a firm in our sample
files for bankrupcty at any time in our sample, then that firm is
excluded for the rest of the sample. Bankrupt firms will remain in
the sample for the quarter in which they file for bankruptcy, but
their last quarterly value will equal their default probability on
their last day before bankruptcy filing. (Analysis\_Y9C.do)
\item We currently do \textbf{not} filter out daily default probability
data just because the probability exceeds some pre-defined threshold.
We pursued this at one point, and decided that this would lead to
removing too many legitimate default probabilities. (Analysis\_Y9C.do)
\item The current benchmark NVI excludes FDIC-insured deposits and foreign
deposits from a firm's inside-system liabilities. (Analysis\_Y9C.do)
\end{itemize}
\item ffunds.do
\begin{itemize}
\item The 'Other' category's assets are made up of the Flow of Funds assets
for ABS issuers, credit unions, funding corporations, and finance
companies (i.e. the remainder after we account for things we can easily
see in KMV)
\end{itemize}
\item compile\_agg\_sector\_data.do
\begin{itemize}
\item While this script includes some code for a potential ``Mutual Fund''
aggregate sector node, that node does not currently factor into the
final NVI calculation. This was done earlier in our process, before
we decided that those assets really ultimately belong to households,
not the financial system. Code is kept around in case we change our
mind, though.
\item Our classifications of financial firms are determined wholly from
the mkmvind\_name\_desc variable. 
\item In our aggregate sector node compilations, we include any companies
incorporated in the US, Bermuda, or Cayman Islands (chartered?)
\item For some firms, total assets do not appear in the KMV dataset every
quarter. In this case, we linearly interpolate between the quarters
we have data.
\item Merrill Lynch has a strange duplicate firm in the KMV dataset with
huge default probabilities and identical assets (MKMVID = N04946).
This ghost firm is deleted in our analysis. 
\item Merrill Lynch also seems to appear in the KMV database for a few quarters
after their assets are assumed by Bank of America. Merrill is removed
for the quarters when this seems to be the case.
\item The 'Other' category's probability of default is currently computed
from the firms whose mkmvind\_name\_desc equals Investment Management,
Finance NEC, or Finance Company. 
\item While the aggregate SIFMA dealers data is used for asset and liability
breakdowns for the Top 10 and Top 11-25 dealer nodes, we still use
KMV data to come up with those nodes' default probabilities. We compile
those numbers based on the asset-weighted average default probabilites
for ``Security Brokers \& Dealers'' that are appropriately ranked
per their assets, as recorded in the KMV balance sheet asset variable
total\_assets\_current\_amt. These values \textbf{include} firms that
are also included in the FR-Y9C sample. 
\item For other aggregate sector nodes (Other, Insurance, REIT), firms that
appear in the FR-Y9C sample are \textbf{excluded} from the calculation
of that aggregate node's average default probability (i.e. their weight
is set to zero). In theory, that firm's assets should also be removed
from the total assets assigned to the aggregate node, although we
are \textbf{not} doing that in the current iteration of the NVI.
\end{itemize}
\end{itemize}

\end{document}
